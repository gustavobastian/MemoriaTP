\chapter{Diseño e implementación} % Main chapter title
\label{Chapter3} 

En este capítulo se explica como se utilizaron las herramientas mencionadas en el capítulo 2 para resolver los objetivos del trabajo presentados en el capítulo 1.

% Change X to a consecutive number; for referencing this chapter elsewhere, use \ref{ChapterX}

\definecolor{mygreen}{rgb}{0,0.6,0}
\definecolor{mygray}{rgb}{0.5,0.5,0.5}
\definecolor{mymauve}{rgb}{0.58,0,0.82}

%%%%%%%%%%%%%%%%%%%%%%%%%%%%%%%%%%%%%%%%%%%%%%%%%%%%%%%%%%%%%%%%%%%%%%%%%%%%%
% parámetros para configurar el formato del código en los entornos lstlisting
%%%%%%%%%%%%%%%%%%%%%%%%%%%%%%%%%%%%%%%%%%%%%%%%%%%%%%%%%%%%%%%%%%%%%%%%%%%%%
\lstset{ %
  backgroundcolor=\color{white},   % choose the background color; you must add \usepackage{color} or \usepackage{xcolor}
  basicstyle=\footnotesize,        % the size of the fonts that are used for the code
  breakatwhitespace=false,         % sets if automatic breaks should only happen at whitespace
  breaklines=true,                 % sets automatic line breaking
  captionpos=b,                    % sets the caption-position to bottom
  commentstyle=\color{mygreen},    % comment style
  deletekeywords={...},            % if you want to delete keywords from the given language
  %escapeinside={\%*}{*)},          % if you want to add LaTeX within your code
  %extendedchars=true,              % lets you use non-ASCII characters; for 8-bits encodings only, does not work with UTF-8
  %frame=single,	                % adds a frame around the code
  keepspaces=true,                 % keeps spaces in text, useful for keeping indentation of code (possibly needs columns=flexible)
  keywordstyle=\color{blue},       % keyword style
  language=[ANSI]C,                % the language of the code
  %otherkeywords={*,...},           % if you want to add more keywords to the set
  numbers=left,                    % where to put the line-numbers; possible values are (none, left, right)
  numbersep=5pt,                   % how far the line-numbers are from the code
  numberstyle=\tiny\color{mygray}, % the style that is used for the line-numbers
  rulecolor=\color{black},         % if not set, the frame-color may be changed on line-breaks within not-black text (e.g. comments (green here))
  showspaces=false,                % show spaces everywhere adding particular underscores; it overrides 'showstringspaces'
  showstringspaces=false,          % underline spaces within strings only
  showtabs=false,                  % show tabs within strings adding particular underscores
  stepnumber=1,                    % the step between two line-numbers. If it's 1, each line will be numbered
  stringstyle=\color{mymauve},     % string literal style
  tabsize=2,	                   % sets default tabsize to 2 spaces
  title=\lstname,                  % show the filename of files included with \lstinputlisting; also try caption instead of title
  morecomment=[s]{/*}{*/}
}


%----------------------------------------------------------------------------------------
%	SECTION 1
%----------------------------------------------------------------------------------------
\section{Generación del entorno base para el desarrollo del sistema de backend}
 En esta sección se presentan las distintas partes que componen el sistema.
 Es necesario aclarar que es un sistema distribuido, por lo que la página web de configuración, el \textit{backend}, el \textit{broker} y las aplicaciones móviles pueden estar en distintos dispositivos fisicamente. La única condicion necesaria es que se encuentren en una red que permite la conexión entre ellos. La red puede ser una red local (implementada con router ) o bien internet.
%La idea de esta sección es resaltar los problemas encontrados, los criterios utilizados y la justificación de las decisiones que se hayan tomado.

%Se puede agregar código o pseudocódigo dentro de un entorno lstlisting con el siguiente código:

%\begin{verbatim}
%\begin{lstlisting}[caption= "un epígrafe descriptivo"]
%	las líneas de código irían aquí...
%\end{lstlisting}
%\end{verbatim}

%A modo de ejemplo:

%\begin{lstlisting}[label=cod:vControl,caption=Pseudocódigo del lazo principal de control.]  % Start your code-block

%#define MAX_SENSOR_NUMBER 3
%#define MAX_ALARM_NUMBER  6
%#define MAX_ACTUATOR_NUMBER 6

%uint32_t sensorValue[MAX_SENSOR_NUMBER];		
%FunctionalState alarmControl[MAX_ALARM_NUMBER];	//ENABLE or DISABLE
%state_t alarmState[MAX_ALARM_NUMBER];						//ON or OFF
%state_t actuatorState[MAX_ACTUATOR_NUMBER];			//ON or OFF

%void vControl() {

%	initGlobalVariables();
	
%	period = 500 ms;
		
%	while(1) {

%		ticks = xTaskGetTickCount();
		
%		updateSensors();
		
%		updateAlarms();
		
%		controlActuators();
		
%		vTaskDelayUntil(&ticks, period);
%	}
%}
%\end{lstlisting}
\section{Broker Mosquitto}

El broker Mosquitto es una parte central del sistema ya que se encarga de distribuir los mensajes entre los distintos elementos del sistema. El único requisito es que se encuentre instalado en la misma red que los clientes (en un sistema que no requiere de muchos recursos). Su instalación, en un entorno operativo Ubuntu se realiza con los siguientes comandos:




\begin{lstlisting}[caption=  Instalación/lanzamiento del broker Mosquitto]
		sudo apt-get install mosquitto mosquitto-clients
		sudo systemctl enable mosquitto.service
\end{lstlisting}

Para configurarlo se debe editar el archivo : "/etc/mosquitto/mosquitto.conf":

\begin{lstlisting}[caption=  Contenido archivo mosquitto.conf]
log_dest file /var/log/mosquitto/mosquitto.log

include_dir /etc/mosquitto/conf.d
listener 9001
protocol websockets
allow_anonymous true

\end{lstlisting}

En este archivo, se especifíca los puertos que se utilizará, en este caso, el puerto 9001 con el protocolo Websockets. Se permite la interconexión de cualquier cliente y el archivo donde se guardará un log de todos los eventos se encuentra en la ubicación "/var/log/mosquitto/mosquitto.log".

\pagebreak
\section{Sistema Docker}

El archivo docker-compose.yml contiene los siguientes servicios instalados:

\begin{itemize}
\item servidor mysql:




\end{itemize}




\section{Base de datos del sistema}
En esta sección se describiran las distintas tablas que se almacenan en la base de datos y son útiles para el sistema. Estas tablas pueden ser observadas y editadas con el frontend phpMyAdmin (pero no es recomendable). 

\begin{enumerate}

\item Tabla usuarios (\textit{User}): tabla que contiene la información de los usuarios del sistema. El campo id se incrementa automáticamente, es decir, al incorporar un nuevo usuario al sistema la base de datos asigna el id (se utiliza este campo como llave primaria).En el campo contraseña se almacena el hash de la misma(generado con Bcrypt \citep{WEBSITE:31}). En el campo ocupación se almacena uno de valores: "Enfermero","Médico" o "Administrador". El campo estado (\textit{State}) se utiliza en versiones posteriores del sistema.

\begin{figure}[ht]
	\centering
	\includegraphics[scale=.6]{./Figures/dB(user).png}
	\caption{Tabla Usuarios.}
	\label{fig:Tabla Usuarios (base de datos)}
\end{figure}

\item Tabla Camas(\textit{Bed}): tabla que contiene la información de las camas como ser su identificación (como en el item anterior se incrementa automáticamente y es clave primaria), ubicación(piso y cuarto) y el número de dispositivo llamador. 

\begin{figure}[ht]
	\centering
	\includegraphics[scale=.6]{./Figures/dB(bed).png}
	\caption{Tabla Camas.}
	\label{fig:Tabla Camas (base de datos)}
\end{figure}

\item Tabla Paciente (\textit{Patient}): tabla que contiene la información de los pacientes. En esta tabla el id no se incrementa automáticamente (ya que puede ser asignado manualmente por el administrador). Ademas, los campos \textbf{bedId}, \textbf{notesTableId} y \textbf{userTableId} contienen claves foraneas que identifican un elemento de la tabla \textit{Bed} (cama), \textit{notesTable} (tablas de notas) y \textit{userTable} (tablas de médicos relacionados al paciente).


\begin{figure}[ht]
	\centering
	\includegraphics[scale=.6]{./Figures/dB(patient).png}
	\caption{Tabla pacientes.}
	\label{fig:Tabla pacientes (base de datos)}
\end{figure}


\item Relación médicos-pacientes : para generar la relación se utilizan un par de tablas intermedias que permiten que un mismo paciente posea varios médicos tratándolo. Además, un paciente puede utilizar a los mísmos médicos de otro paciente... es una de las ventajas de las bases de datos relacionales.


\begin{figure}[ht]
	\centering
	\includegraphics[scale=.45]{./Figures/tabla-medicos-pacientes.png}
	\caption{Relación médicos-pacientes.}
	\label{fig:Relación médicos-pacientes (base de datos)}
\end{figure}



\item Relación tabla de notas- pacientes: como en el caso anterior, se utiliza una tabla de notas generales con una tabla intermedia que los indexa. En este caso y como es particular de cada paciente no se puede repetir.


\begin{figure}[ht]
	\centering
	\includegraphics[scale=.45]{./Figures/patient-notes.png}
	\caption{Relación notas-pacientes.}
	\label{fig:Relación notas-pacientes (base de datos)}
\end{figure}

\pagebreak

\item Tabla de eventos programados: En esta tabla se cargan las tareas programadas para un paciente. El identificador se incrementa al generarse, el nro de paciente es enviado por el cliente, el campo tipo de evento puede ser \textbf{diario},\textbf{mensual} o \textbf{diario} y el campo datetime a la hora que se requiere que se realice. El campo nota es donde se coloca la tarea a realizar (Por ejemplo, suministrar X gramos de medicamento Y).


\begin{figure}[ht]
	\centering
	\includegraphics[scale=.45]{./Figures/Events.png}
	\caption{Tabla de eventos programados.}
	\label{fig:Tabla de eventos programados (base de datos)}
\end{figure}


\item Tabla de log eventos: En esta tabla se guardan los eventos relacionados con un paciente. El id se asigna al guardarse el evento. El tipo puede ser \textbf{tarea programada} o \textbf{llamada paciente}. El id del paciente, el id de usuario y la nota se cargan por el usuario que finalizó la accion.


\begin{figure}[ht]
	\centering
	\includegraphics[scale=.45]{./Figures/logEvents.png}
	\caption{Tabla de registro de eventos.}
	\label{fig:Tabla de registro de eventos (base de datos)}
\end{figure}

\item Tabla de especialidades (\textit{SpecTable}): En esta tabla se las distintas especialidades que poseen todos los enfermeros. 

\item Tabla de especialidades de enfermeros (\textit{NurseSpecTable}): En esta tabla se relacionan a los enfermeros con su Id y las distintas especialidades que pueden realizar. Un enfermero puede poseer más de una especialidad.

\item Tabla de tratamientos de pacientes(\textit{PatientSpecTable}): En esta tabla se relacionan a los pacientes con su Id y una tratamiento(se obtiene de la SpecTable). 

\item Tabla de códigos QR (\textit{QRbed}): En esta tabla se almacenan en texto los códigos QR correspondientes a las camas para su reconocimiento.

\item Tabla de prioridades (\textit{PriorityTable}): En esta tabla el administrador puede asignar prioridades a las distintas camas del sistema, las prioridades son de 5 niveles, siendo 5 la más alta prioridad.



\end{enumerate}



\section{Sistema de gestión}

Se diseñó el sistema backend fragmentándolo en tres partes:

\begin{itemize}
\item Monitoreo: clases que ayudan a publicar a todos los clientes MQTT los estados del sistema (usuario logeados y estados de habitaciones/pacientes) 
\item Respuestas al cliente página web: expone una API para que los administradores del sistema puedan incorporar usuarios,pacientes,camas ... observar estadísticas, etc.
\item Respuestas al cliente MQTT: se subscribe a los tópicos correspondientes para responder consultas.
\end{itemize}

En la figura \ref{fig:Estructura de directorio(backend)} se presenta la estructura de directorios del backend.

\begin{figure}[ht]
	\centering
	\includegraphics[scale=.45]{./Figures/projectStructure.png}
	\caption{Estructura de directorio (backend).}
	\label{fig:Estructura de directorio(backend)}
\end{figure}

La descripción de los contenidos de cada carpeta es:
\begin{itemize}
\item middleware: contiene el programa que filtra el acceso a la api desde el front end web.
\item Monitoring: contiene clases con información del estado del sistema que se actualiza de en tiempo real. Cada una de estas clases contiene una lista de todos los elementos y se presenta a los tópicos correspondientes.
\item mqtt: contiene clases que se encargan de procesar y responder los mensajes que arriban por medio del protocolo mqtt.
\item mysql: contiene la clase con el pool de conexiones a la base de datos.
\item routes: contiene las rutas de express para poder obtener acceso a la base de datos desde http.
\end{itemize}

\subsection{Descripción de las clases para monitoreo del sistema}

Las siguientes son clases que se utilizan para reportar mediante MQTT los estados del sistema. Para resumir se presentan los elementos principales, pero no las funciones que las componen.

\begin{itemize}

\item \textbf{ Monitoreo de camas:}\\
Se genera una clase \textit{Bedslist} la cual contiente una lista de elementos json con información de cada cama.

\begin{lstlisting}[caption=  Clase Bedlist]
class  BedsList  {    
    constructor() {
         this.bedlist=[{id:0,st:0,spec:0}];                        
    }
...}
\end{lstlisting}

Los componentes de cada elemento son 
\begin{enumerate}
\item id: identificador de cama.
\item st: estado de la cama. Puede poseer los siguientes estados: 
\begin{itemize}
\item 0: no ocupada.
\item 1: ocupada.
\item 2: llamando paciente.
\item 3: llamada aceptada por enfermero.
\item 4: enfermero atendiendo.
\item 5: tarea programada.
\item 6: enfermero solicitando ayuda.
\end{itemize}
\item spec: número de tratamiento del paciente en la cama (es utilizado por los clientes enfermeros para filtrar si pueden o no atenderlo).
\end{enumerate}

\item \textbf{ Monitoreo de usuarios:}\\
Se genera una clase \textit{UserList} la cual contiente una lista de elementos json con información de cada usuario.

\begin{lstlisting}[caption=  Clase Userlist]
constructor() {
         this.UserList=[{id:0,st:0}];                
        
    }
...}
\end{lstlisting}

Los componentes de cada elemento son 
	\begin{enumerate}
		\item id: identificador de usuario.
		\item st: estado de usuario. Puede poseer los siguientes estados: 
			\begin{itemize}
				\item 0: no logeado.
				\item 1: logeado.
			\end{itemize}
	\end{enumerate}




\item \textbf{ Monitoreo de eventos programados:}\\
Esta clase se utiliza para presentar a los clientes las notas de tareas programadas que se lanzan. Es una lista que se vacía cuando la tarea finaliza (sirve para mantener ordenadas las tareas programadas).

\begin{lstlisting}[caption=  Clase CalendarList]
constructor() {
         this.CalendarList=[{calendarId:0,bedId:0,note:null}];                
        
    }
...}
\end{lstlisting}

Los componentes de cada elemento son 
	\begin{enumerate}
		\item calendarId: identificador de evento.
		\item bedId: número de cama.
		\item note: Nota de la tarea(obtenida de la base de datos).		

	\end{enumerate}




\item \textbf{ Monitoreo de paciente-usuario-tipo de evento:}\\
Esta clase se utiliza para en todo momento saber quien está atendiendo a un paciente.

\begin{lstlisting}[caption=  Clase BedsUserList]
 constructor() {
         this.beduserlist=[{bedId:0,userId:0,type:0}];                        
    }
...}
\end{lstlisting}

Los componentes de cada elemento son 
	\begin{enumerate}
		\item bedId: número de cama.
		\item userId: número de usuario.
		\item type: tipo de evento: 1 llamador, 2 calendario.		

	\end{enumerate}


\end{itemize}
\subsection{Descripción de API MQTT para la mensajería de la aplicación móvil}

El archivo principal es mqtt.js (dentro de la carpeta mqtt) y contiene la inicialización de la conexión inicial al broker, el lanzamiento de las tareas de monitoreo (publicación de las listas antes definidas en los tópicos específicos) y la escucha de los mensajes. Una vez recibido un mensaje, dependiendo del tipo del mismo se deriva los mismo a la clase correspondiente que realiza el tratamiento.


\begin{lstlisting}[caption=  Tareas ejecutadas por mqtt.js]
var client = mqtt.connect(process.env.MQTT_CONNECTION)
//listening to  messages
client.on('connect', function () {
  client.subscribe('/User/#', function (err) {
    if (err) {
    console.log("error:"+err);
    }
  })

  client.subscribe('/Pacient/#', function (err) {
    if (err) {      
      console.log("error:"+err);
    }
  })
  client.subscribe('/Beds/#', function (err) {
    if (err) {      
      console.log("error:"+err);
    }
  })

  //task that will publish beds state each second
  setInterval(publishBedStates, 10000);
  //task that will publish users state each second
  setInterval(publishUserStates, 15000);
})
\end{lstlisting}

Como se puede observar en el código anterior, el sistema recibe mensajería por medio de tres tópicos principales User, Pacient y Beds. Por otra parte, se utilizan variables de entorno como ser \textbf{MQTT\textunderscore CONNECTION}  , que contiene informacion de la IP del broker y el puerto del mismo.

Por último, se llama a 2 funciones publishBedStates y publishUserStates que publican los listados de los estados de camas y usuarios cada cierto tiempo respectivamente.


En general, contenido útil de los mensajes MQTT tiene la siguiente forma:

\begin{lstlisting}[caption=  Formato mensaje MQTT]

payload={_username: "xxxx",_content: "xxx", _bedId: xx, _type: x}

\end{lstlisting}

La descripción de los campos es la siguiente:
\begin{itemize}
\item \textunderscore username: nombre del usuario que envió el comando
\item \textunderscore content: contenido
\item \textunderscore bedId: numero de cama a la que se hace referencia
\item \textunderscore type: tipo de mensaje
\end{itemize}

El tipo de mensaje permite al sistema derivar a las clases correspondientes las tareas a realizar. 

%\begin{verbatim}
\begin{table}[h]
	\centering
	\caption[Tipos de mensajes MQTT]{Tipos de mensajes del sistema.}
	\begin{tabular}{l c c}    
		\toprule
		\textbf{Tipo}     & \textbf{Descripción} \\
		\midrule
		1 & Login.    \\		
		2 & Logout.\\
		3 & Escribir nota a paciente.\\
		4 & Solicitar información de paciente de una cama.\\
		5 & Solicitar notas del paciente. \\
		7 & Mensaje de texto entre usuarios.\\
		8 & Solicitar información de cama (ubicación).\\
		9 & Solicitando información de camas de cada médico.\\
		10 & Solicitar información de un paciente de una cama.\\
		11 & Chequeo de QR.\\
		12 & Aceptación de parte de una enfermera.\\
		13 & Finalización de trabajo.\\
		14 & Solicitar ayuda.\\
		16 & Especialización de enfermera.\\
		17 & Consultar tabla de médicos asignada a paciente.\\
		18 & Eliminar nota de paciente.\\
		19 & Cancelar visita.\\
		20 & Notas de evento calendario.\\
		22 & Mensaje de audio entre usuarios.\\
		\bottomrule
		\hline
	\end{tabular}
	\label{tab:Modelo}
\end{table}
%\end{verbatim}

Las distintas clases que se utilizan son: 
\begin{enumerate}
\item beds: consulta a la base de datos información relacionada a las camas.
\item patient: consulta a la base de datos información relacionada a los pacientes. 
\item users: consulta a la base de datos información relacionada a los usuarios (funciones de logueo y deslogueo).
\item calendar: consulta a la base de datos información relacionada a los eventos programados.
\item nurse: consulta a la base de datos información sobre las especialidades de los enfermeros.
\end{enumerate}

Todas estas clases responden en un tópico correspondiente a los clientes que consultan.

\pagebreak
Las excepciones al formato general de los mensajes vienen dadas por:
\begin{itemize}
\item Mensaje de último testamento: informa que un usuario se desconectó. En su \textit{payload} contiene información del nombre de usuario y se publica en el tópico $"/User/Disconnect"$.
\item Mensaje de llamador: informa que un dispositivo llamador fue accionado por un paciente. En su \textit{payload} contiene información del número de llamador y se publica en el tópico $"/Beds/Caller-events"$.

\end{itemize}


\pagebreak

\subsection{Descripción de API REST para aplicación Web}

La API del trabajo utiliza Express junto con Cookie-parser. 
 
En este trabajo se utiliza un \textit{middleware} (capa de software intermedia entre los recursos y la consulta de los usuarios), el cual consiste en una función que realiza el control de acceso a los endpoints. Por definición la página web de configuración solo puede ser accedida por los usuarios administradores, para lograr autenticación el \textit{backend} utiliza las librería \texbf{jsonwebtoken} \citep{WEBSITE:32} y \textbf{bcrypt} \citep{WEBSITE:31}. 

Se denomina ruteo a la forma que un \textit{endpoint} (punto de acceso, en español) de una aplicación responde a las peticiones de los clientes. En las aplicaciones express, el objeto express posee métodos que realizan las operaciones sobre la base de datos o el sistema.

\begin{lstlisting}[caption=  Rutas express]
app.use('/api/patient',auth.isAuthenticated,routerPatient);
app.use('/api/user',auth.isAuthenticated,routerUser);
app.use('/api/messages',auth.isAuthenticated,routerMessages);
app.use('/api/notes',auth.isAuthenticated,routerNotes);
app.use('/api/beds',auth.isAuthenticated,routerBeds);
app.use('/api/usersTable',auth.isAuthenticated,routerUsersTable);
app.use('/api/medicalTable',auth.isAuthenticated,routerMedicalTable);
app.use('/api/QR',auth.isAuthenticated,routerQR);
app.use('/api/events',auth.isAuthenticated,routerEvents);
app.use('/api/logEvents',auth.isAuthenticated,routerLogEvents);
app.use('/api/Statistics',auth.isAuthenticated,routerStatistics);
app.use('/api/authentication',routerAuthenticate);
app.use('/api/specTable',auth.isAuthenticated,routerSpecTable);
app.use('/api/nurseSpecTable',auth.isAuthenticated,routerNurseSpecTable);
app.use('/api/treatment',auth.isAuthenticated,routerPatientSpecTable);
\end{lstlisting}


La ruta authentication se utiliza para entregar un \textit{token} al cliente, en dicha función se verifica que el usuario sea administrador.

\begin{lstlisting}[caption=  Logueo web]
routerAuthenticate.post('/', async function(req, res) {
    if (req.body) {
        var user = req.body;
        await pool.query('Select username,password,occupation from User WHERE username=?',[user.username], async function(err, result, fields) {
            if (err) {
                var response = JSON.stringify(response_conform);
                res.status(403).send({
                    errorMessage: 'Auth required!'});
                return;    
            }
            else{
                try{
                testUser.username=result[0].username;
                testUser.password=result[0].password;
                }catch (e){res.status(403).send({
                    errorMessage: 'Auth required!'});
                    return;    
                    }
                    await bcrypt.compare(user.password, result[0].password, (err, resultComp)=> {

                        if ((resultComp==true  ) &&(result[0].occupation=="Administrador") ){
                            var token = jwt.sign(user, process.env.JWT_SECRET,{
                                expiresIn: process.env.JWT_EXP_TIM
                            });
                            res.status(200).send({
                                signed_user: result[0],
                                token: token
                            });
                        } else {res.status(403).send({
                                errorMessage: 'Auth required!'
                            }); }
                        })                    
                }      
            }) 
        } else {
            res.status(403).send({
                errorMessage: 'Please provide username and password'
            });
       }

    })


\end{lstlisting}


Luego, la función \textit{auth.isAuthenticated(request,response,next)}, dentro del archivo \textit{./middleware/authentication} filtra el acceso solo a la página web con un usuario administrador logueado.
\begin{lstlisting}[caption=  Control de token]
exports.isAuthenticated = async(req, res,next)=> {
    let authHeader = (req.headers.authorization || '');
    if (authHeader.startsWith("Bearer ")) {
        token = authHeader.substring(7, authHeader.length);
    } else {
        return res.send({ message: 'No Auth' });
    }
    if (token) {
        jwt.verify(token, process.env.JWT_SECRET, function(err) {
            if (err) {
                console.log("Alguien cambio el token, no es valido");
                return res.json({ message: 'Invalid Token' });
            } else {
                console.log("Validado el token y todo ok");
                return next();
            }
        });
    } else {
        return res.send({ message: 'No token' });
    }
}
\end{lstlisting}

Cada recurso posee su ruta correspondiente. Esta forma de organizar el código permite que se incorporen funcionalidades al \textit{backend} de forma sencilla.

\pagebreak

\section{Página Web(frontend)}

La página web de administración consiste en un \textit{dashboard} (tablero de control) que permite al usuario administrador gestionar el sistema.
Es implementada utilizando el framework Ionic/Angular y las principales áreas de la visualización se observa en la figura \ref{fig:Página web}.

\begin{figure}[ht]
	\centering
	\includegraphics[scale=.48]{./Figures/pagina-web.png}
	\caption{Página web.}
	\label{fig:Página web}
\end{figure}

En el area menú se presentan las distintas utilizades de la página:
\begin{itemize}
\item Login: acceso al sistema. El visitante ingresa su nombre de usuario y contraseña y se le otorga los permisos correspondientes (la conexión recibe el \textit{token} del \textit{backend}).
\item MQTT: permite editar o probar la conexión al broker MQTT del sistema. Es necesario si se desea monitorear el estado de los usuarios o camas en tiempo real.
\item Monitoreo: se visualiza el estado de las camas o de los usuarios en tiempo real.
\item Camas: permite editar información de las camas como ser código QR, numero de llamador, cuarto y piso.
\item Usuarios: permite agregar,editar o dar de baja usuarios.
\item Pacientes: permite agregar,editar o dar de baja pacientes.
\item Log Eventos: permite observar el listado de eventos en cada paciente sucedidos.
\item Calendario: permite observar o agregar tareas rutinarias asignadas a un paciente.
\item Estadísticas: permite observar el numero de intervenciones de un enfermero, la distribucion de especialidades dentro del grupo de enfermeros y la distribución de tratamientos requeridos por los pacientes. Con esta información el administrador puede notificar sobre necesidades de mayor capacitación en un tema para los enfermeros.
\end{itemize}


\subsection{Estructura y organización del software}
\subsection{Acceso de usuario}
\subsection{Monitoreo del sistema}
Para poder acceder a las funcionalidades es necesario logearse con las credenciales de un administrador. Si se quiere monitorear al sistema en tiempo real, es necesario configurar la ubicación del \textit{broker} dentro de la solapa MQTT. El monitoreo de camas se observa en la figura \ref{fig:Monitoreo de camas}.

\begin{figure}[ht]
	\centering
	\includegraphics[scale=.40]{./Figures/monitoreo-camas.png}
	\caption{Monitoreo de camas.}
	\label{fig:Monitoreo de camas}
\end{figure} 
 
\subsection{Gestión de tareas programadas}
\subsection{Estadísticas del sistema}
\section{Aplicación Móvil}
\subsection{Estructura y organización del software}
\subsection{Configuración del broker y acceso de usuario}
\subsection{Modo administrador}
\subsection{Modo médico}
\subsection{Modo enfermera}
\subsection{Interacción médico-enfermera}
\section{Contraste con los requerimientos}

