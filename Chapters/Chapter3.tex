\chapter{Diseño e implementación} % Main chapter title
\label{Chapter3} 

En este capítulo se explica la forma en que se utilizaron las herramientas mencionadas en el capítulo 2 para resolver los objetivos del trabajo presentados en el capítulo 1.

% Change X to a consecutive number; for referencing this chapter elsewhere, use \ref{ChapterX}

\definecolor{mygreen}{rgb}{0,0.6,0}
\definecolor{mygray}{rgb}{0.5,0.5,0.5}
\definecolor{mymauve}{rgb}{0.58,0,0.82}

%%%%%%%%%%%%%%%%%%%%%%%%%%%%%%%%%%%%%%%%%%%%%%%%%%%%%%%%%%%%%%%%%%%%%%%%%%%%%
% parámetros para configurar el formato del código en los entornos lstlisting
%%%%%%%%%%%%%%%%%%%%%%%%%%%%%%%%%%%%%%%%%%%%%%%%%%%%%%%%%%%%%%%%%%%%%%%%%%%%%
\lstset{ %
  backgroundcolor=\color{white},   % choose the background color; you must add \usepackage{color} or \usepackage{xcolor}
  basicstyle=\footnotesize,        % the size of the fonts that are used for the code
  breakatwhitespace=false,         % sets if automatic breaks should only happen at whitespace
  breaklines=true,                 % sets automatic line breaking
  captionpos=b,                    % sets the caption-position to bottom
  commentstyle=\color{mygreen},    % comment style
  deletekeywords={...},            % if you want to delete keywords from the given language
  %escapeinside={\%*}{*)},          % if you want to add LaTeX within your code
  %extendedchars=true,              % lets you use non-ASCII characters; for 8-bits encodings only, does not work with UTF-8
  %frame=single,	                % adds a frame around the code
  keepspaces=true,                 % keeps spaces in text, useful for keeping indentation of code (possibly needs columns=flexible)
  keywordstyle=\color{blue},       % keyword style
  language=[ANSI]C,                % the language of the code
  %otherkeywords={*,...},           % if you want to add more keywords to the set
  numbers=left,                    % where to put the line-numbers; possible values are (none, left, right)
  numbersep=5pt,                   % how far the line-numbers are from the code
  numberstyle=\tiny\color{mygray}, % the style that is used for the line-numbers
  rulecolor=\color{black},         % if not set, the frame-color may be changed on line-breaks within not-black text (e.g. comments (green here))
  showspaces=false,                % show spaces everywhere adding particular underscores; it overrides 'showstringspaces'
  showstringspaces=false,          % underline spaces within strings only
  showtabs=false,                  % show tabs within strings adding particular underscores
  stepnumber=1,                    % the step between two line-numbers. If it's 1, each line will be numbered
  stringstyle=\color{mymauve},     % string literal style
  tabsize=2,	                   % sets default tabsize to 2 spaces
  title=\lstname,                  % show the filename of files included with \lstinputlisting; also try caption instead of title
  morecomment=[s]{/*}{*/}
}


%----------------------------------------------------------------------------------------
%	SECTION 1
%----------------------------------------------------------------------------------------
\section{Generación del entorno base para el desarrollo del sistema de backend}
\label{Generación del entorno base para el desarrollo del sistema de backend}
 En esta sección se presentan las distintas partes que componen el sistema.
La solución fue concebida como un sistema distribuido, por lo que la página web de configuración, el backend, el \textit{broker} y las aplicaciones móviles pueden estar en distintos dispositivos físicamente. La única condición necesaria es que se encuentren en una red que permite la conexión entre ellos. La red puede ser una red local (implementada con router) o bien internet.
La aplicación web utiliza HTTP para interactuar con el backend y las aplicaciones móviles utilizan MQTT.
%La idea de esta sección es resaltar los problemas encontrados, los criterios utilizados y la justificación de las decisiones que se hayan tomado.

%Se puede agregar código o pseudocódigo dentro de un entorno lstlisting con el siguiente código:

%\begin{verbatim}
%\begin{lstlisting}[caption= "un epígrafe descriptivo"]
%	las líneas de código irían aquí...
%\end{lstlisting}
%\end{verbatim}

%A modo de ejemplo:

%\begin{lstlisting}[label=cod:vControl,caption=Pseudocódigo del lazo principal de control.]  % Start your code-block

%#define MAX_SENSOR_NUMBER 3
%#define MAX_ALARM_NUMBER  6
%#define MAX_ACTUATOR_NUMBER 6

%uint32_t sensorValue[MAX_SENSOR_NUMBER];		
%FunctionalState alarmControl[MAX_ALARM_NUMBER];	//ENABLE or DISABLE
%state_t alarmState[MAX_ALARM_NUMBER];						//ON or OFF
%state_t actuatorState[MAX_ACTUATOR_NUMBER];			//ON or OFF

%void vControl() {

%	initGlobalVariables();
	
%	period = 500 ms;
		
%	while(1) {

%		ticks = xTaskGetTickCount();
		
%		updateSensors();
		
%		updateAlarms();
		
%		controlActuators();
		
%		vTaskDelayUntil(&ticks, period);
%	}
%}
%\end{lstlisting}


\section{Broker Mosquitto}
\label{Broker Mosquitto}

El broker Mosquitto es una parte central del sistema ya que se encarga de distribuir los mensajes entre los distintos elementos. El único requisito es que se encuentre instalado en la misma red que los clientes (en un dispositivo que no requiere de muchos recursos). Su instalación, en un entorno operativo Ubuntu se realiza con los siguientes comandos:


\begin{lstlisting}[caption=  Instalación/lanzamiento del broker Mosquitto.]
		sudo apt-get install mosquitto mosquitto-clients
		sudo systemctl enable mosquitto.service
\end{lstlisting}

Para configurarlo se debe editar el archivo: "/etc/mosquitto/mosquitto.conf", donde se especifica el puerto que se utiliza, en este caso, el puerto 9001 con el protocolo Websockets. Se permite la interconexión de cualquier cliente y el archivo donde se guarda el log de todos los eventos se encuentra en la ubicación "/var/log/mosquitto/mosquitto.log".

\begin{lstlisting}[label=cod:mosquitto.conf,caption=  Contenido archivo mosquitto.conf.]
log_dest file /var/log/mosquitto/mosquitto.log

include_dir /etc/mosquitto/conf.d
listener 9001
protocol websockets
allow_anonymous true

\end{lstlisting}



\pagebreak
\section{Sistema Docker}

Para iniciar el contenedor \textit{Docker} se ejecuta el comando: ''\textit{Docker-compose up}''.  Este comando busca en el archivo docker-compose.yml la información de los distintos servicios que debe inicializar y su configuración.

Los siguientes servicios son instalados:

\begin{itemize}
\item servidor mysql: se utiliza la imagen: 5.7. Utiliza el puerto 3036 para las comunicaciones. 
\item servidor mysql-admin: se utiliza la imagen: phpmyadmin/phpmyadmin. Se configura el puerto 8001:80 para servir la aplicación.
\item backend node: se utiliza la imagen abassi/nodejs-server:10.0-dev.
\end{itemize}

Por otra parte, se configura una red interna NurseSystem-net que hace de \textit{bridge} (puente) con la red del sistema host.



\section{Base de datos del sistema}
En esta sección se describen las distintas tablas que se almacenan en la base de datos y son útiles para el sistema.  

\begin{enumerate}

\item Tabla usuarios (\textit{User}): tabla que contiene la información de los usuarios del sistema. El campo id se incrementa automáticamente, es decir, al incorporar un nuevo usuario al sistema la base de datos asigna el id (se utiliza este campo como llave primaria). En el campo contraseña se almacena el hash de la misma (generado con Bcrypt \citep{WEBSITE:31}). El campo ocupación representa la actividad a la que se dedica el usuario en el establecimiento y puede tomar uno de los siguientes valores: ''Enfermero'', ''Médico'' o ''Administrador''. El campo estado (\textit{State}) se utiliza en versiones posteriores del sistema. La estructura de la tabla se presenta en la figura \ref{fig:Tabla Usuarios (base de datos)}.

\begin{figure}[ht]
	\centering
	\includegraphics[scale=.70]{./Figures/dB(user).png}
	\caption{Tabla Usuarios.}
	\label{fig:Tabla Usuarios (base de datos)}
\end{figure}

\item Tabla Camas (\textit{Bed}): tabla que contiene la siguiente información de las camas: su identificación (como en el ítem anterior se incrementa automáticamente y es clave primaria), ubicación (piso y cuarto) y el número de dispositivo llamador. La estructura de la tabla se presenta en la figura \ref{fig:Tabla Camas (base de datos)}.

\begin{figure}[ht]
	\centering
	\includegraphics[scale=.70]{./Figures/dB(bed).png}
	\caption{Tabla Camas.}
	\label{fig:Tabla Camas (base de datos)}
\end{figure}
\pagebreak
\item Tabla Paciente (\textit{Patient}): tabla que contiene la información de los pacientes. En esta tabla el id no se incrementa automáticamente (ya que es asignado manualmente por el administrador). Además, los campos \textit{bedId}, \textit{notesTableId} y \textit{userTableId} contienen claves foráneas que identifican un elemento de la tabla
\textit{Bed} (cama), \textit{notesTable} (tablas de notas) y \textit{userTable} (tablas de médicos relacionados al paciente). La estructura de la tabla se presenta en la figura \ref{fig:Tabla pacientes (base de datos)}.


\begin{figure}[ht]
	\centering
	\includegraphics[scale=.70]{./Figures/dB(patient).png}
	\caption{Tabla pacientes.}
	\label{fig:Tabla pacientes (base de datos)}
\end{figure}


\item Relación médicos-pacientes: para generar la relación se utilizan un par de tablas intermedias que permiten que un mismo paciente posea varios médicos asignados. Además, un paciente puede utilizar a los mismos médicos de otro paciente. Se presenta la relación en la figura \ref{fig:Relación médicos-pacientes}.
\begin{figure}[ht]
	\centering
	\includegraphics[scale=.65]{./Figures/tabla-medicos-pacientes.png}
	\caption{Relación médicos-pacientes.}
	\label{fig:Relación médicos-pacientes}
\end{figure}
\item Relación tabla de notas-pacientes: como en el caso anterior, se utiliza una tabla de notas generales con una tabla intermedia que los indexa. En este caso, y como es particular de cada paciente, no se puede repetir.  Se presenta la relación en la figura \ref{fig:Relación notas-pacientes (base de datos)}.
\begin{figure}[ht]
	\centering
	\includegraphics[scale=.65]{./Figures/patient-notes.png}
	\caption{Relación notas-pacientes.}
	\label{fig:Relación notas-pacientes (base de datos)}
\end{figure}


\pagebreak




\item Tabla de eventos programados: En esta tabla se cargan las tareas programadas para un paciente. Se representa en la figura \ref{fig:Tabla de eventos programados (base de datos)} y sus campos son los siguientes:
\begin{itemize}
\item eventId: identificador de evento, se auto incrementa.
\item patientId: número de paciente, es asignado por el cliente y hace referencia a la identificación del paciente
\item type: tipo de evento, puede ser \textit{diario}, \textit{mensual} o \textit{anual} 
\item Datetime: momento que se requiere que se realice la acción.
\item Nota: es donde se coloca la tarea a realizar (Por ejemplo, suministrar X gramos de medicamento Y).
\end{itemize}


\begin{figure}[ht]
	\centering
	\includegraphics[scale=.70]{./Figures/Events.png}
	\caption{Tabla de eventos programados.}
	\label{fig:Tabla de eventos programados (base de datos)}
\end{figure}

\item Tabla de log eventos: En esta tabla se guardan los eventos relacionados con un paciente. Su estructura se muestra en \ref{fig:Tabla de registro de eventos (base de datos)} y sus componentes son:
\begin{itemize}
\item logEventId: se asigna al guardarse el evento.
\item type: el tipo puede ser \textit{tarea programada} o \textit{llamada paciente}. 
\item pacientId: identificador del paciente. 
\item userId: identificador del usuario que finalizó la tarea. 
\item Nota: en el caso de tarea programada, se guarda la nota de la tarea.
\item Nota2: se cargan por el usuario que finalizó la acción al escribir la memoria de lo realizado.
\end{itemize}



\begin{figure}[ht]
	\centering
	\includegraphics[scale=.70]{./Figures/logEvents.png}
	\caption{Tabla de registro de eventos.}
	\label{fig:Tabla de registro de eventos (base de datos)}
\end{figure}

\item Tabla de especialidades (\textit{SpecTable}): En esta tabla se presentan las distintas especialidades que poseen los enfermeros. Cada especialidad es un tratamiento que puede ser asignado a un paciente.

\item Tabla de especialidades de enfermeros (\textit{NurseSpecTable}): En esta tabla se relacionan a los enfermeros con su Id y las distintas especialidades que pueden realizar. Un enfermero puede poseer más de una especialidad.

\item Tabla de tratamientos de pacientes (\textit{PatientSpecTable}): En esta tabla se relacionan a los pacientes con su Id y un tratamiento (se obtiene de la SpecTable). 

\item Tabla de códigos QR (\textit{QRbed}): En esta tabla se almacenan, en formato texto, los códigos QR correspondientes a las camas para su reconocimiento.

\item Tabla de prioridades (\textit{PriorityTable}): En esta tabla el administrador puede asignar prioridades a las distintas camas del sistema.  Las prioridades son de 5 niveles, siendo 5 la más alta prioridad.



\end{enumerate}



\section{Sistema de gestión}

El backend se diseñó fragmentándolo en tres partes:

\begin{itemize}
\item Módulo de monitoreo: clases que ayudan a publicar a todos los clientes MQTT los estados del sistema (usuario logeados y estados de habitaciones/pacientes). 
\item Módulo de respuestas al cliente página web: expone una API para que los administradores del sistema puedan incorporar usuarios, pacientes, camas, observar estadísticas, etc.
\item Módulo de respuestas al cliente MQTT: se subscribe a los tópicos correspondientes para responder consultas.
\end{itemize}

En la figura \ref{fig:Estructura de carpetas (backend)} se presenta la estructura de directorios del backend.

\begin{figure}[ht]
	\centering
	\includegraphics[scale=.45]{./Figures/projectStructure.png}
	\caption{Estructura de carpetas (backend).}
	\label{fig:Estructura de carpetas (backend)}
\end{figure}

La descripción de los contenidos de cada carpeta es:
\begin{itemize}
\item middleware: contiene el programa que filtra el acceso a la API desde  la página web.
\item Monitoring: contiene clases con información del estado del sistema que se actualiza cada cierto tiempo. Cada una de estas clases contiene una lista de todos los elementos y se presenta a los tópicos correspondientes.
\item mqtt: contiene clases que se encargan de procesar y responder los mensajes que se reciben por medio del protocolo MQTT.
\item mysql: contiene la clase con el \textit{pool} (''grupo'') de conexiones a la base de datos. El objetivo es poder reutilizar conexiones por distintos usuarios.
\item routes: contiene las rutas de express para acceder a la base de datos desde las peticiones por HTTP.
\end{itemize}

\pagebreak
\subsection{Descripción de las clases para monitoreo del sistema}

Las siguientes son las clases que se utilizan para reportar mediante MQTT los estados del sistema. Para resumir se presentan los elementos principales, pero no las funciones que las componen.

\begin{itemize}

\item \textit{ Monitoreo de camas} (código \ref{cod:bedlist}):\\
La clase \textit{Bedslist} contiene una lista de elementos JSON con información de cada cama (se instancia al iniciar el backend).

\begin{lstlisting}[label=cod:bedlist,caption=  Clase Bedlist.]
class  BedsList  {    
    constructor() {
         this.bedlist=[{id:0,st:0,spec:0}];                        
    }
...}
\end{lstlisting}

Los componentes de cada elemento son 
\begin{enumerate}
\item id: identificador de cama.
\item st: estado de la cama. Los valores que puede poseer este campo son: 
\begin{itemize}
\item 0: no ocupada.
\item 1: ocupada.
\item 2: llamando paciente.
\item 3: llamada aceptada por enfermero.
\item 4: enfermero atendiendo.
\item 5: tarea programada.
\item 6: enfermero solicitando ayuda.
\end{itemize}
\item spec: número de tratamiento del paciente en la cama (es utilizado por los clientes enfermeros para filtrar si pueden o no atenderlo).
\end{enumerate}


\item \textit{ Monitoreo de usuarios} (código \ref{cod:Userlist}):\\
La clase\textit{UserList} contiene una lista de elementos JSON con información de cada usuario (se instancia al iniciar el backend):

\begin{lstlisting}[label=cod:Userlist,caption=  Clase Userlist.]
class  UserList  {   
constructor() {
         this.UserList=[{id:0,st:0}];                
        
    }
...}
\end{lstlisting}

Los componentes de cada elemento son 
	\begin{enumerate}
		\item id: identificador de usuario.
		\item st: estado de usuario. Puede poseer los siguientes estados: 
			\begin{itemize}
				\item 0: no logeado.
				\item 1: logeado.
			\end{itemize}
	\end{enumerate}




\item \textit{ Monitoreo de eventos programados} (código \ref{cod:Calendarlist}):\\
La clase se utiliza para presentar a los clientes las notas de las tareas programadas que se lanzan. Es una lista que se vacía cuando la tarea finaliza (sirve para mantener ordenadas las tareas programadas).

\begin{lstlisting}[label=cod:Calendarlist,caption=  Clase CalendarList.]
class  Calendarlist  {   
constructor() {
         this.CalendarList=[{calendarId:0,bedId:0,note:null}];                
        
    }
...}
\end{lstlisting}

Los componentes de cada elemento son 
	\begin{enumerate}
		\item calendarId: identificador de evento.
		\item bedId: número de cama.
		\item note: Nota de la tarea (obtenida de la base de datos).		

	\end{enumerate}



\pagebreak
\item \textit{ Monitoreo de paciente-usuario-tipo de evento} (código \ref{cod:BedUserlist}):\\
La clase BedUserlist se utiliza para saber quien está atendiendo, en todo momento, a un paciente.

\begin{lstlisting}[label=cod:BedUserlist,caption=  Clase BedsUserList.]
class  BedUserlist  { 
 constructor() {
         this.beduserlist=[{bedId:0,userId:0,type:0}];                        
    }
...}
\end{lstlisting}

Los componentes de cada elemento son 
	\begin{enumerate}
		\item bedId: número de cama.
		\item userId: número de usuario.
		\item type: tipo de evento: 
		\begin{itemize}
			\item provocado por un paciente
			\item visita programada		
		\end{itemize}		 	

	\end{enumerate}


\end{itemize}
\subsection{Descripción de API MQTT para la mensajería de la aplicación móvil}

En esta sección se describe los módulos que permiten la interacción del sistema con los clientes utilizando el protocolo MQTT.


El módulo principal es mqtt.js (dentro de la carpeta mqtt) y contiene la inicialización de la conexión inicial al broker, el lanzamiento de las tareas de monitoreo (publicación de las listas antes definidas en los tópicos específicos) y la escucha de los mensajes. Una vez recibido un mensaje, dependiendo del tipo, se deriva a la clase correspondiente que realiza el tratamiento acorde a la petición. En el fragmento de código  \ref{cod:Tareas MQTT} se presentan las tareas de inicialización del módulo.



\begin{lstlisting}[label=cod:Tareas MQTT,caption=  Tareas ejecutadas por mqtt.js.]
var client = mqtt.connect(process.env.MQTT_CONNECTION)
//listening to  messages
client.on('connect', function () {
  client.subscribe('/User/#', function (err) {
    if (err) {
    console.log("error:"+err);
    }
  })

  client.subscribe('/Pacient/#', function (err) {
    if (err) {      
      console.log("error:"+err);
    }
  })
  client.subscribe('/Beds/#', function (err) {
    if (err) {      
      console.log("error:"+err);
    }
  })







  //task that will publish beds state each second
  setInterval(publishBedStates, 10000);
  //task that will publish users state 
  setInterval(publishUserStates, 15000);
})
\end{lstlisting}

El sistema recibe mensajería por medio de tres tópicos principales: \textit{User}, \textit{Pacient} y \textit{Beds}, según se observa en las líneas 4, 10 y 15 de \ref{cod:Tareas MQTT}. Por otra parte, se utilizan variables de entorno como ser \textit{MQTT\textunderscore CONNECTION}, que contiene información de la IP del broker y el puerto.

Por último, dentro de esta función se setean dos temporizadores para ejecutar las funciones publishBedStates y publishUserStates que publican los listados de los estados de camas y usuarios cada cierto tiempo.




El contenido útil (\textit{payload}) de los mensajes MQTT tiene la siguiente forma:

\begin{lstlisting}[label=cod:Mensaje MQTT,caption=  Formato mensaje MQTT.]

payload={_username: "xxxx",_content: "xxx", _bedId: xx, _type: x}

\end{lstlisting}

La descripción de los campos es la siguiente:
\begin{itemize}
\item \textunderscore username: nombre del usuario que envió el comando.
\item \textunderscore content: contenido.
\item \textunderscore bedId: número de cama a la que se hace referencia.
\item \textunderscore type: tipo de mensaje.
\end{itemize}

El tipo de mensaje permite al sistema derivar las tareas a realizar a las clases correspondientes. En la tabla \ref{tab:Tipos de Mensajes MQTT} se presentan la asignación numérica de cada tipo de mensaje:

%\begin{verbatim}
\begin{table}[h]
	\centering
	\caption[Tipos de mensajes MQTT]{Tipos de mensajes del sistema.}
	\begin{tabular}{l c c}    
		\toprule
		\textbf{Tipo}     & \textbf{Descripción} \\
		\midrule
		1 & Login.    \\		
		2 & Logout.\\
		3 & Escribir nota a paciente.\\
		4 & Solicitar información de paciente de una cama.\\
		5 & Solicitar notas del paciente. \\
		7 & Mensaje de texto entre usuarios.\\
		8 & Solicitar información de cama (ubicación).\\
		9 & Solicitando información de camas de cada médico.\\
		10 & Solicitar información de un paciente de una cama.\\
		11 & Chequeo de QR.\\
		12 & Aceptación de parte de una enfermera.\\
		13 & Finalización de trabajo.\\
		14 & Solicitar ayuda.\\
		16 & Especialización de enfermera.\\
		17 & Consultar tabla de médicos asignada a paciente.\\
		18 & Eliminar nota de paciente.\\
		19 & Cancelar visita.\\
		20 & Notas de evento calendario.\\
		22 & Mensaje de audio entre usuarios.\\
		\bottomrule
		\hline
	\end{tabular}
	\label{tab:Tipos de Mensajes MQTT}
\end{table}
%\end{verbatim}

Las distintas clases que se utilizan son: 
\begin{enumerate}
\item beds: consulta a la base de datos información relacionada a las camas.
\item patient: consulta a la base de datos información relacionada a los pacientes. 
\item users: consulta a la base de datos información relacionada a los usuarios (funciones de logueo y deslogueo).
\item calendar: consulta a la base de datos información relacionada a los eventos programados.
\item nurse: consulta a la base de datos información sobre las especialidades de los enfermeros.
\end{enumerate}

Todas estas clases responden en un tópico correspondiente a los clientes que consultan.

Las excepciones al formato general de los mensajes vienen dadas por:
\begin{itemize}
\item Mensaje de último testamento: informa que un usuario se desconectó. En su \textit{payload} contiene información del nombre de usuario y se publica en el tópico $"/User/Disconnect"$.
\item Mensaje de llamador: informa que un dispositivo llamador fue accionado por un paciente. En su \textit{payload} contiene información del número de llamador y se publica en el tópico $"/Beds/Caller-events"$.

\end{itemize}


\subsection{Descripción de API Rest para aplicación Web}
\label{Descripción de API Rest para aplicación Web}

La API desarrollada utiliza Express junto con Cookie-parser. 
 
En este trabajo se utiliza un \textit{middleware} (capa de software intermedia entre los recursos y la consulta de los usuarios), el cual consiste en una función que realiza el control de acceso a los endpoints. Para lograr la autenticación el backend utiliza las librería \textit{jsonwebtoken} \citep{WEBSITE:32} y \textit{bcrypt} \citep{WEBSITE:31}. En una primera versión, solo se habilita el uso de la página de administración a usuarios administradores.

Se denomina ruteo a la forma que una aplicación express direcciona a las peticiones del cliente a los respectivos módulos que se encargan de su tratamiento. El objeto express posee métodos que realizan las operaciones sobre la base de datos o el sistema. El fragmento de código con las rutas express se observa en el código \ref{cod:Rutas_Express}:

\begin{lstlisting}[label=cod:Rutas_Express,caption=  Rutas express.]
app.use('/api/patient',auth.isAuthenticated,routerPatient);
app.use('/api/user',auth.isAuthenticated,routerUser);
app.use('/api/messages',auth.isAuthenticated,routerMessages);
app.use('/api/notes',auth.isAuthenticated,routerNotes);
app.use('/api/beds',auth.isAuthenticated,routerBeds);
app.use('/api/usersTable',auth.isAuthenticated,routerUsersTable);
app.use('/api/medicalTable',auth.isAuthenticated,routerMedicalTable);
app.use('/api/QR',auth.isAuthenticated,routerQR);
app.use('/api/events',auth.isAuthenticated,routerEvents);
app.use('/api/logEvents',auth.isAuthenticated,routerLogEvents);
app.use('/api/Statistics',auth.isAuthenticated,routerStatistics);
app.use('/api/authentication',routerAuthenticate);
app.use('/api/specTable',auth.isAuthenticated,routerSpecTable);
app.use('/api/nurseSpecTable',auth.isAuthenticated,  routerNurseSpecTable);
app.use('/api/treatment',auth.isAuthenticated,routerPatientSpecTable);
\end{lstlisting}

Brevemente, a continuación, se presenta la funcionalidad de cada endpoint:
\begin{itemize}
\item /api/patient: endpoint que permite la gestión de pacientes (alta, baja, modificación)
\item /api/user: endpoint que permite la gestión de usuarios (alta, baja, modificación)
\item /api/messages: permite obtener un listado de los mensajes entre usuarios (en esta versión de código y debido las limitaciones de almacenamiento del servidor prototipo, se encuentra comentada toda esta funcionalidad)
\item /api/notes: endpoint que permite la gestión de las notas de los pacientes (alta, baja, modificación). 
\item /api/beds: endpoint que permite la gestión de las camas (alta, baja y modificación). 
\item /api/usersTable: endpoint que permite la gestión de las tablas de usuarios. Son las agrupaciones de los médicos que pueden atender a un paciente. 
\item /api/medicalTable: endpoint que permite la gestión de las tablas de tablas de usuarios.
\item /api/QR: enpoint que permite la gestión de los códigos QR (alta, baja y modificación).
\item /api/events: endpoint que permite ingresar una tarea programada al calendario.
\item /api/logEvents: endpoint que permite obtener el listado de eventos.
\item /api/Statistics: endpoint que permite obtener información estadística de la base de datos. Por ejemplo:
\begin{enumerate}
\item número total de visitas de cada enfermero, en formato JSON.
\item número total de llamados de cada paciente, en formato JSON.
\item número de pacientes con igual tratamiento, en formato JSON.
\item número de enfermeras con una especialización, en formato JSON.
\end{enumerate}

Un ejemplo de uso de esta información puede ser: al observarse un incremento de pacientes con cierto tratamiento se puede solicitar capacitar a nuevas enfermeras en esa especialización. Por el contrario, si disminuye un tratamiento se capacita al personal o bien se lo reduce.

\item /api/authentication: endpoint que permite loguearse al sistema. Se explica con más detalle por su importancia.
\item /api/specTable: endpoint que permite modificar la tabla de especialidades de enfermeros (tratamientos para los pacientes). Se permite agregar o quitar especialidades.
\item /api/nurseSpecTable: endpoint que permite asignar o quitar especialidades a una enfermera. Por ejemplo, una enfermera recibió una capacitación y desde ese momento queda habilitada para atender pacientes con una cierta necesidad.
\item /api/treatment: endpoint que permite asignar a  un paciente un tratamiento o bien modificar el tratamiento.

\end{itemize}

La ruta authentication se utiliza para entregar un \textit{token} al cliente. En dicha función se verifica que el usuario sea administrador.

\begin{lstlisting}[label=cod:Logueo Web,caption=  Logueo web.]
routerAuthenticate.post('/', async function(req, res) {
    if (req.body) {
        var user = req.body;
        await pool.query('Select username,password,occupation from User WHERE username=?',[user.username], async function(err, result, fields) {
            if (err) {
                var response = JSON.stringify(response_conform);
                res.status(403).send({
                    errorMessage: 'Auth required!'});
                return;    
            }
            else{
                try{
                testUser.username=result[0].username;
                testUser.password=result[0].password;
                }catch (e){res.status(403).send({
                    errorMessage: 'Auth required!'});
                    return;    
                    }
                    await bcrypt.compare(user.password, result[0].password, (err, resultComp)=> {

                        if ((resultComp==true  ) &&(result[0].occupation=="Administrador") ){
                            var token = jwt.sign(user, process.env.JWT_SECRET,{
                                expiresIn: process.env.JWT_EXP_TIM
                            });
                            res.status(200).send({
                                signed_user: result[0],
                                token: token
                            });
                        } else {res.status(403).send({
                                errorMessage: 'Auth required!'
                            }); }
                        })                    
                }      
            }) 
        } else {
            res.status(403).send({
                errorMessage: 'Please provide username and password'
            });
       }

    })

\end{lstlisting}

En el código \ref{cod:Logueo Web} se presenta la función que se encarga de autenticar al usuario. Al recibir un mensaje de autenticación, el backend genera la siguiente secuencia de comprobación:
\begin{enumerate}
\item Verificación que el usuario existe: línea 4
\item Verificación que el password recibido corresponde con el usuario: línea 21
\item Verificación que el usuario es administrador: línea 21
\end{enumerate}

Luego de eso genera un token que expira luego de un cierto tiempo dado por una variable de entorno y se responde al frontend.

Una vez logueado, la función \textit{auth.isAuthenticated(request,response,next)}, dentro del archivo \textit{./middleware/authentication} comprueba que el usuario que solicita el recurso posea el token correspondiente.
\begin{lstlisting}[label=cod: Autorización,caption=  Control de token.]
exports.isAuthenticated = async(req, res,next)=> {
    let authHeader = (req.headers.authorization || '');
    if (authHeader.startsWith("Bearer ")) {
        token = authHeader.substring(7, authHeader.length);
    } else {
        return res.send({ message: 'No Auth' });
    }
    if (token) {
        jwt.verify(token, process.env.JWT_SECRET, function(err) {
            if (err) {
                console.log("Alguien cambio el token, no es valido");
                return res.json({ message: 'Invalid Token' });
            } else {
                console.log("Validado el token y todo ok");
                return next();
            }
        });
    } else {
        return res.send({ message: 'No token' });
    }
}
\end{lstlisting}

Cada recurso posee su ruta correspondiente. Esta forma de organizar el código permite que se incorporen funcionalidades al backend de forma sencilla.

\pagebreak

\section{Página Web}

La página web de administración consiste en un \textit{dashboard} (tablero de control) que permite al usuario administrador gestionar el sistema.
Fue implementada utilizando el framework Ionic/Angular, con el lenguaje de programación Typescript y las principales secciones de la visualización se observan en la figura \ref{fig:Página web}.

\begin{figure}[ht]
	\centering
	\includegraphics[scale=.40]{./Figures/pagina-web.png}
	\caption{Página web.}
	\label{fig:Página web}
\end{figure}

En el área menú se presentan las distintas utilidades de la página:
\begin{itemize}
\item Login: acceso al sistema. El visitante ingresa su nombre de usuario y contraseña y se le otorga los permisos correspondientes (la conexión recibe el \textit{token} del backend).
\item MQTT: permite editar o probar la conexión al broker MQTT del sistema. Es necesario si se desea monitorear el estado de los usuarios o camas en tiempo real.
\item Monitoreo: se visualiza el estado de las camas o de los usuarios en tiempo real.
\item Camas: permite editar información de las camas como ser código QR, número de llamador, cuarto y piso.
\item Usuarios: permite agregar, editar o dar de baja usuarios.
\item Pacientes: permite agregar, editar o dar de baja pacientes.
\item Log Eventos: permite observar el listado de eventos que se hayan guardado en la base de datos.
\item Calendario: permite observar o agregar tareas rutinarias asignadas a un paciente.
\item Estadísticas: permite observar el número de intervenciones de un enfermero, la distribución de especialidades dentro del grupo de enfermeros y la distribución de tratamientos requeridos por los pacientes. Con esta información el administrador puede notificar sobre necesidades de mayor capacitación en un tema para los enfermeros.
\end{itemize}


\subsection{Estructura y organización del software}

El software web generado contiene dos versiones: una para un entorno de desarrollo y otra para un entorno productivo. En este desarrollo, las características del entorno se encuentran en una carpeta \textit{''/src/environments''}. 

Para ejecutar la aplicación en la máquina local se debe ejecutar el comando:  \textit{'' ionic lab ''}, mientras que para compilar la página productiva se debe ejecutar \textit{'' ionic build prod ''}. El resultado de la construcción se encuentra dentro de la carpeta \textit{''/www''}.

Todo el código de la aplicación se encuentra en la carpeta \textit{''/src/app''} y se presenta una captura en la figura \ref{fig:Carpetas página web.}.


\begin{figure}[ht]
	\centering
	\includegraphics[scale=.60]{./Figures/codigoFront.png}
	\caption{Estructura de carpetas de la aplicación web.}
	\label{fig:Carpetas página web.}
\end{figure}


Dentro de la carpeta \textit{models} se encuentran las distintas clases que se utilizan para la comunicación con el backend. El listado de clases, con una pequeña mención de su utilidad, se presenta a continuación:

\begin{itemize}
\item bed-status: gestiona información del estado de las camas  (''ocupada'', ''llamando'',...) y del tratamiento que utiliza el paciente.
\item bed: gestiona información de las camas.
\item calendarEvent: se utiliza para gestionar información de tareas programas.
\item logEvent: se utiliza para gestionar información de eventos realizados.
\item medicalTable: relaciona un usuario con una tabla de médicos.
\item message-model: modelo de mensaje que se transmite a través de MQTT.
\item note: modelo de nota para un paciente (no utilizado en la aplicación web).
\item nurseSpec: clase que se utiliza para observar las especialidades de los distintos enfermeros (número, número de especialidad y nombre de especialidad).
\item patient: clase que contiene información del paciente (id, nombre, apellido, cama, índice en la tabla de notas, índice en la tabla de médicos)
\item patientTreat: clase que relaciona un paciente con un tratamiento (índice, número de paciente, número de tratamiento y nombre de tratamiento).
\item qr: relaciona un índice con un texto (se utiliza para guardar o recuperar índices de qr)
\item spec: clase que contiene una especialidad o tratamiento. Contiene un índice y un texto con el nombre.
\item user-status: clase que permite obtener desde el sistema el estado de los usuarios (identificación de usuario y estado ''logueado'' o ''no logeado'').
\item user: contiene información del usuario (nombre, apellido, ocupación, contraseña, nombre de usuario).
 
\end{itemize}


Dentro de la carpeta \textit{pages} se encuentran las distintas secciones de acción de la página (se presentan en el área de visualización).

Dentro de la carpeta \textit{services} se organizan los distintos servicios.

Dentro de la carpeta \textit{folder} se encuentra el layout principal de la página.

La aplicación posee un módulo principal llamado ''app.module''. Utilizando el angular-router, se redirecciona desde cualquier página a una deseada y se presenta dentro del campo de visualización de la pantalla principal. 



\subsection{Acceso de usuario}

Un usuario no puede acceder a las distintas utilidades de la página si no se encuentra logeado o si no es administrador. Todas las consultas al backend necesitan poseer un \textit{token} de autenticación en su encabezado. Esto se logra mediante el servicio interceptor:


\begin{lstlisting}[label=cod:Inserción Token,caption=  Inserción de token.]
export class AuthInterceptorService implements HttpInterceptor {

  constructor(private _router:Router) { }

  intercept(req: HttpRequest<any>, next: HttpHandler): Observable<HttpEvent<any>> {
    if (req.url.includes("/authenticate")){
      return next.handle(req);
    }
    const token: string = localStorage.getItem('token');
    let request = req;
	if (token) {
      request = this.addToken(request, token);
      return next.handle(request)
      .pipe(catchError((error: HttpErrorResponse) => {
        let errorMsg = '';
        if (error.error instanceof ErrorEvent) {
          console.log('Client Error');
          errorMsg = `Error: ${error.error.message}`;
        }
        else {
          console.log('Server Error');
          errorMsg = `Error Code: ${error.status},  Message: ${error.message}`;
        }
        console.log(errorMsg);
        return throwError(errorMsg);
      })
      );
    }else{
      return next.handle(request)
      //if i have no token navigate to login
      this._router.navigate(['/login']);
    }    
  }

  private addToken(request: HttpRequest<any>, token: string) {
    return request.clone({
      setHeaders: {
        'Authorization': `Bearer ${token}`
      }
    });
  }
}

\end{lstlisting}

\pagebreak

\subsection{Monitoreo del sistema}
Si se quiere monitorear al sistema, es necesario configurar la ubicación del \textit{broker} dentro de la solapa MQTT. El monitoreo de camas se observa en la figura \ref{fig:Monitoreo de camas.}. El sistema reporta el estado de las camas cada un segundo (o cuando hay un cambio abrupto) y el estado de los usuarios cada un segundo y quinientos milisegundos.

\begin{figure}[ht]
	\centering
	\includegraphics[scale=.55]{./Figures/monitoreo-camas.png}
	\caption{Monitoreo de camas.}
	\label{fig:Monitoreo de camas.}
\end{figure} 

\subsection{Gestión de tareas programadas}

Si se desea generar un nuevo evento programado, se selecciona del menú  calendario, se selecciona el número de paciente y luego se presiona agregar. En ese momento se presenta el formulario de la figura \ref{fig:Tareas programadas.}, se lo completa y se presiona enviar. Los datos a completar son:
\begin{itemize}
\item Nota: campo donde se presenta la tarea propiamente dicha.
\item tipo: se selecciona si es diario, semanal o mensual.
\item en el calendario se selecciona fecha y hora de inicio, desde ese momento se repite según el tipo.
\end{itemize}

\begin{figure}[ht]
	\centering
	\includegraphics[scale=.40]{./Figures/tarea-programada.png}
	\caption{Tareas programadas.}
	\label{fig:Tareas programadas.}
\end{figure} 

El backend guarda en la base de datos la nueva tarea y genera la acción correspondiente.

\pagebreak
\subsection{Estadísticas del sistema}

En esta sección se puede observar la cantidad de atenciones por enfermero guardadas en la base de datos, la relación entre las distintas especialidades y la relación entre tratamientos y pacientes. En la figura \ref{fig:Relación paciente tratamiento.} se muestra una de las subsecciones que se pueden seleccionar. Para generar las gráficas se utiliza HighCharts \citep{WEBSITE:33} (las especialidades son de fantasía, creadas solo para el ejemplo). 
\begin{figure}[ht]
	\centering
	\includegraphics[scale=.65]{./Figures/web/pacientes-Tratamiento.png}
	\caption{Relación paciente tratamiento.}
	\label{fig:Relación paciente tratamiento.}
\end{figure} 


\pagebreak
\section{Aplicación Móvil}
\label{Aplicación Móvil}
En esta sección se presenta la funcionalidad de la aplicación móvil y algunos detalles de la implementación.
\label{Estructura y organización del software}
\subsection{Estructura y organización del software}
La estructura de directorios del proyecto es la presentada en la figura \ref{fig: Estructura de código fuente de la aplicación móvil.}.

\begin{figure}[ht]
	\centering
	\includegraphics[scale=.60]{./Figures/app/estructura-app.png}
	\caption{ Estructura de carpetas de la aplicación móvil.}
	\label{fig: Estructura de código fuente de la aplicación móvil.}
\end{figure} 

El código de las distintas páginas por las que navega el usuario se encuentra dentro de \textit{''./src/app/pages''}. Los modelos (''clases'') utilizados se encuentra en la carpeta \textit{''./src/app/models''} y los servicios en la carpeta \textit{''./src/app/services''}. La carpeta \textit{''./src/app/chat''} contiene la primera versión la intercomunicación entre enfermeros-médicos que simulaba una sala de reunión  (similar a la ofrecida por el producto en el mercado). Luego se migró a la definitiva, que cumple estrictamente con lo solicitado en un principio, pero por solicitud del cliente, no se eliminó la carpeta.

Como se mencionó en la sección \ref{Generación del entorno base para el desarrollo del sistema de backend}, la aplicación móvil interactúa con el backend solo utilizando websockets MQTT.

La aplicación posee un módulo principal llamado ''app.module''. Utilizando el angular-router, se redirecciona desde cualquier página a una deseada. En el código \ref{cod:Rutas aplicación Móvil.} se presenta como se redireccionan las páginas en Ionic:

\begin{lstlisting}[label=cod:Rutas aplicación Móvil.,caption=  Fragmento de las rutas de la aplicación móvil.]
const routes: Routes = [
  {
    path: 'home',
    loadChildren: () => import('./pages/home/home.module').then( m => m.HomePageModule)
  },
  {
    path: '',
    redirectTo: 'home',
    pathMatch: 'full'
  },
  {
    path: 'mqtt-config',
    loadChildren: () => import('./pages/mqtt-config/mqtt-config.module').then( m => m.MqttConfigPageModule)
  },
  {
    path: 'login',
    loadChildren: () => import('./pages/login/login.module').then( m => m.LoginPageModule)
  },
  ...

\end{lstlisting} 

\pagebreak

\subsection{Configuración del broker y acceso de usuario}

Cuando se inicia la aplicación se presenta una pantalla con dos pulsadores, uno redirige a la página de configuración del broker MQTT y otro a la página de ingreso (ver figura \ref{fig_0:1_de_3}).
%\begin{figure}[ht]
%	\centering
%	\includegraphics[scale=.80]{./Figures/app/inicioApp.png}
%	\caption{ Pantalla inicial, configuración y acceso de la aplicación.}
%	\label{fig: Pantalla inicial, configuración y acceso de la aplicación.}
%\end{figure} 



\begin{figure}[!htpb]
     \centering
     \begin{subfigure}[b]{0.3\textwidth}
         \centering
         \includegraphics[width=1.1\textwidth]{./Figures/app/main-page.png}
         \caption{Pantalla Inicial.}
         \label{fig_0:1_de_3}
     \end{subfigure}
     \hfill
     \begin{subfigure}[b]{0.3\textwidth}
         \centering
         \includegraphics[width=1.1\textwidth]{app/login-page.png}
         \caption{Pantalla login.}
         \label{fig_0:2_de_3}
     \end{subfigure}
     \hfill
     \begin{subfigure}[b]{0.3\textwidth}
         \centering
         \includegraphics[width=1.1\textwidth]{./Figures/app/system-config.png}
         \caption{Configuración.}
         \label{fig_0:3_de_3}
     \end{subfigure}
        \caption{Inicio de sistema.}
        \label{fig:Pantalla inicial, configuración y acceso de la aplicación.}
\end{figure}

En la página de configuración (figura \ref{fig_0:3_de_3}) se ingresa la IP del broker del sistema y el puerto. Luego se puede probar la comunicación y guardar los parámetros en el localStorage del dispositivo, como se presenta en el código \ref{cod:LocalStorage}.

\begin{lstlisting}[label=cod:LocalStorage,caption=  Funciones del servicio que guardan en el localStorage.]
  /**
   * Saving port values to localStorage
   */
  public saveValues = async () => {     
    this.localSto.saveValuesString('MQTTSERVER',this.MQTTSERVER);
    this.localSto.saveValuesNumber('MQTTPORT',this.MQTTPORT);
  };
\end{lstlisting}

En la página de logueo (figura \ref{fig_0:2_de_3}) se ingresan el nombre de usuario, y su contraseña.

La aplicación publica la información en el tópico ''/User/username'' con el formato de mensaje correspondiente y recibe en ''/Session/nrodesesión'' el número de usuario y el modo de uso. De esta manera la aplicación se setea en el modo correspondiente.



\subsection{Modo administrador}
En el modo administrador (figura \ref{fig: Pantallas de administración.}) se puede monitorear el estado de los usuarios y las camas. Para seleccionar lo que se desea monitorear, se presiona el botón correspondiente.

\begin{figure}[ht]
	\centering
	\includegraphics[scale=.70]{./Figures/app/administracion.png}
	\caption{ Pantallas de administración.}
	\label{fig: Pantallas de administración.}
\end{figure} 

El orden en que se presentan las camas tienen que ver con la prioridad que se les asignó a cada una.



\pagebreak
\subsection{Modo médico}
Al ingresar en modo médico, la aplicación queda en modo de espera hasta que el usuario decida que hacer. El diagrama de estados se presenta en la figura \ref{fig: Diagrama de estados en modo médico.} y las distintas pantallas a las que se puede acceder se presentan en la figura \ref{fig:Modo médico}. 

\begin{figure}[ht]
	\centering
	\includegraphics[scale=.65]{./Figures/app/modo-medico.png}
	\caption{ Diagrama de estados en modo médico.}
	\label{fig: Diagrama de estados en modo médico.}
\end{figure} 

Las acciones que puede realizar un médico son:
\begin{itemize}
\item Actualizar una nota de un paciente: para la gestión de estos mensajes se utiliza el tópico ''/Pacient/id'' donde id es el número de paciente. Cuando el médico solicita información del paciente (nombre, apellido y número) se responde  en ''/Pacient/id/info''. Cuando se consulta las notas se responde en ''/Pacient/id/notes'' y cuando se quiere ingresar una nueva nota se publica en ''/Pacient/id/newNote''. La pantalla correspondiente se muestra en la figura \ref{fig_1:2de3}.
\item Responder a una consulta de una enfermera. La pantalla correspondiente se muestra en la figura \ref{fig_1:3de3}
\item Monitorear las camas con los pacientes que le fueron encargados: simplemente filtra por sus pacientes el listado publicado en ''/Beds/status''. La pantalla correspondiente se muestra en la figura \ref{fig_1:1de3}

\end{itemize}


\begin{figure}[!htpb]
     \centering
     \begin{subfigure}[b]{0.3\textwidth}
         \centering
         \includegraphics[width=.95\textwidth]{./Figures/app/doctor-patient-1.png}
         \caption{Monitoreo de pacientes.}
         \label{fig_1:1de3}
     \end{subfigure}
     \hfill
     \begin{subfigure}[b]{0.3\textwidth}
         \centering
         \includegraphics[width=.95\textwidth]{app/doctor-notes5.png}
         \caption{Gestión de notas.}
         \label{fig_1:2de3}
     \end{subfigure}
     \hfill
     \begin{subfigure}[b]{0.3\textwidth}
         \centering
         \includegraphics[width=.95\textwidth]{./Figures/app/doctor-message-audio.png}
         \caption{Mensajes.}
         \label{fig_1:3de3}
     \end{subfigure}
        \caption{Modo médico.}
        \label{fig:Modo médico}
\end{figure}

\pagebreak
\subsection{Modo enfermera}

Es el modo más complejo por la cantidad de opciones que se pueden presentar.

Al ingresar en modo enfermera, la aplicación consulta al backend por la especialización del enfermero correspondiente y la respuesta se obtiene de escuchar en el tópico ''/User/userId/Specs'' donde userId se recibió al loguearse.

Para el usuario, la aplicación queda en modo espera hasta que se reciba una notificación de necesidad de atención. El diagrama de estados se presenta en la figura \ref{fig: Diagrama de estados en modo enfermera.}.

 

La aplicación se encuentra escuchando al tópico MQTT ''/Beds/status'', y filtra los estados (solo acepta estados con llamadas, llamadas programadas o ayuda) y por especialidad del enfermero. De esta manera, cuando un enfermero acepta una tarea, automáticamente se actualiza el backend y ningún otro enfermero puede aceptarla.



\begin{figure}[ht]
	\centering
	\includegraphics[scale=.60]{./Figures/app/estados-enf.png}
	\caption{ Diagrama de estados en modo enfermera.}
	\label{fig: Diagrama de estados en modo enfermera.}
\end{figure} 



Cuando arriba la notificación, se presenta una tarjeta con dos botones: información de la cama y aceptación. En caso que la enfermera desee consultar donde se encuentra (piso y habitación) debe presionar el botón información. En caso que decida ir a la habitación, debe presionar aceptar.
Al presionar aceptar automáticamente el estado de la cama cambia a desplazándose a la habitación (la información proviene del sistema que recibió la aceptación y cambió el estado de la habitación). Se presentan estas capturas en la figura \ref{fig_2:Recepción de tarea}.

\begin{figure}[!htpb]
     \centering
     \begin{subfigure}[b]{0.3\textwidth}
         \centering
         \includegraphics[width=.95\textwidth]{./Figures/app/enfermera-peticion.png}
         \caption{Se recibe petición.}
         \label{fig_2:1de3}
     \end{subfigure}
     \hfill
     \begin{subfigure}[b]{0.3\textwidth}
         \centering
         \includegraphics[width=.95\textwidth]{app/enfermera-cama.png}
         \caption{Solicita información.}
         \label{fig_2:2de3}
     \end{subfigure}
     \hfill
     \begin{subfigure}[b]{0.3\textwidth}
         \centering
         \includegraphics[width=.95\textwidth]{./Figures/app/yendo-enfermera.png}
         \caption{Tarea aceptada.}
         \label{fig_2:3de3}
     \end{subfigure}
        \caption{Recepción de tarea}
        \label{fig_2:Recepción de tarea}
\end{figure}

Cuando el usuario se encuentra frente al paciente, puede ingresar el código correspondiente a la cama manualmente o bien leer el código QR con el celular, como se observa en la figura \ref{fig: Captura de QR en la aplicación.}.

\begin{figure}[ht]
	\centering
	\includegraphics[scale=.05]{./Figures/app/capturaQR.jpg}
	\caption{ Captura de QR en la aplicación.}
	\label{fig: Captura de QR en la aplicación.}
\end{figure} 


Una vez que se aceptó desde el sistema el código de la cama, en la aplicación se presenta una pantalla con distintos botones (ver figuras \ref{fig_3:Ejecución de tarea}). Las acciones que se permiten realizar son:


\begin{enumerate}
\item Cancelar: envía al sistema automáticamente la solicitud de cancelar la tarea. El sistema vuelve a colocar a la cama en situación de llamada.
\item Listo: envía al sistema automáticamente la solicitud de finalizar la tarea. El sistema coloca a la cama en situación de ocupada.
\item Notas: consulta al sistema las notas referidas al paciente. 
\item Mensajes: permite enviar audio o texto a un médico. 
\item Ayuda: solicita al sistema que marque a la cama como con solicitud de ayuda para que otros enfermeros puedan socorrer al usuario. 
\item Memoria: Permite incorporar un texto que se almacena junto con la tarea al presionar Listo.
\end{enumerate}

\begin{figure}[!htpb]
     \centering
     \begin{subfigure}[b]{0.3\textwidth}
         \centering
         \includegraphics[width=.95\textwidth]{./Figures/app/enfermera-trabajando.png}
         \caption{Menu dentro de habitación.}
         \label{fig_3:1de3}
     \end{subfigure}
     \hfill
     \begin{subfigure}[b]{0.3\textwidth}
         \centering
         \includegraphics[width=.95\textwidth]{app/enfermera-consultaNotas.png}
         \caption{Solicita información y notas.}
         \label{fig_3:2de3}
     \end{subfigure}
     \hfill
     \begin{subfigure}[b]{0.3\textwidth}
         \centering
         \includegraphics[width=.95\textwidth]{./Figures/app/enfermera-msg.png}
         \caption{Abriendo menú de mensajes.}
         \label{fig_3:3de3}
     \end{subfigure}
        \caption{Ejecución de tarea.}
        \label{fig_3:Ejecución de tarea}
\end{figure}
