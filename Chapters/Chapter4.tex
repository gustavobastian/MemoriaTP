% Chapter Template

\chapter{Ensayos y resultados} % Main chapter title
\label{Chapter4} % Change X to a consecutive number; for referencing this chapter elsewhere, use \ref{ChapterX}

En este capítulo se describen los ensayos realizados, se comentan las herramientas que se utilizaron y se presentan los resultados. Se fragmenta el capítulo en tres partes: pruebas unitarias, integración del sistema y contraste contra requerimientos.

%----------------------------------------------------------------------------------------
%	SECTION 1
%----------------------------------------------------------------------------------------


\section{Pruebas unitarias}
En esta sección se presentan las herramientas que se utilizaron para ensayar las distintas partes del sistema en forma separada.

\subsection{Pruebas del servidor Mosquitto}
Para simular mensajes y observar el comportamiento de cada una de las partes del sistema se utilizó la herramienta MQTT explorer   \citep{WEBSITE:29} que se presenta en la figura \ref{fig:MQTT Explorer}.

\begin{figure}[ht]
	\centering
	\includegraphics[scale=.30]{./Figures/mqtt-explorer.png}
	\caption{Imagen de MQTT Explorer.}
	\label{fig:MQTT Explorer}
\end{figure}

Esta herramienta fue importante no solo para realizar ensayos sino también para el desarrollo de las funcionalidades propiamente dichas.


\subsection{Pruebas unitarias de la API Rest}

Para simular el logeo y la consulta a la base de datos se utilizó la herramienta PostMan \citep{WEBSITE:30}, Newman \citep{WEBSITE:35} y la consola de administrador de phpMyAdmin mencionada en \ref{subsec: Bases de datos relacionales}. En la figura \ref{fig:Logueo en el sistema con Postman} se observa el \textit{token} devuelto por el backend al loguearse con las credenciales correspondientes y en \ref{fig:Rechazo Logueo en el sistema con Postman} se observa la respuesta al ingresar una contraseña inadecuada.


En esta secuencia, el cliente (en este caso Postman), envía en el body el nombre de usuario (''username'') y la contraseña (''password'') en formato JSON. Según lo explicado en la sección  \ref{Descripción de API Rest para aplicación Web}, el backend realiza ciertas verificaciones y devuelve un token de sesión en el body.

\begin{figure}[ht]
	\centering
	\includegraphics[scale=.35]{./Figures/auth.png}
	\caption{Logueo en el sistema con Postman.}
	\label{fig:Logueo en el sistema con Postman}
\end{figure}

\begin{figure}[ht]
	\centering
	\includegraphics[scale=.35]{./Figures/no-auth.png}
	\caption{Rechazo de logueo.}
	\label{fig:Rechazo Logueo en el sistema con Postman}
\end{figure}

Postman permite automatizar los test utilizando colecciones (que son secuencias de consultas a la API). Las colecciones se pueden exportar a un formato JSON de modo de poder ser utilizadas por otras herramientas. 

En un entorno de integración continua y entrega continua, una herramienta de testeo automático permite, una vez definidas las colecciones de test, volver a ejecutarlas luego de haber realizado una modificación,  asegurar que no se modificó lo que ya funcionaba y que las mejoras pasan los nuevos test. La ventaja de utilizar newman radica en que al tener un cliente de consola se puede generar un script automático.

Para utilizar Newman, que es el cliente de consola, se deben exportar las colecciones y luego utilizar la línea de comandos con dicha colección (ver figura \ref{fig:Colección logueo usuario en Postman}). Existen varias opciones de reportes de resultados, en este trabajo se utiliza cli y htmlextra, ejecutando en la consola el código \ref{cod:Newman}.

\begin{figure}[ht]
	\centering
	\includegraphics[scale=.35]{./Figures/Postman.png}
	\caption{Colección test de logueo.}
	\label{fig:Colección logueo usuario en Postman}
\end{figure}

\begin{lstlisting}[label=cod:Newman,caption=  Ejecución de Newman en consola.]
>>newman run ./logueo_usuario.postman_collection.json -r cli,htmlextra
\end{lstlisting}

En la figura \ref{fig:reporte cli Newman} se observa el reporte por consola.
\begin{figure}[ht]
	\centering
	\includegraphics[scale=.50]{./Figures/newman-1.png}
	\caption{Reporte en consola de Newman.}
	\label{fig:reporte cli Newman}
\end{figure}




\pagebreak

\section{Integración del sistema}
\label{Integración del sistema}

En esta sección se explica como se instaló el backend y el servidor web en una máquina remota, se instaló la aplicación en múltiples dispositivos, se desarrolló un dispositivo que simula los eventos de los llamadores de los hospitales, se ensayó el sistema durante una semana.  Finalmente se analizan los resultados de la simulación de uso.

Se utilizó como máquina remota una instancia en Amazon Web Services. Los dispositivos móviles en los cuales se instaló la aplicación poseían el sistema Android como sistema operativo.

\subsection{Instalación del sistema en una instancia de AWS}
Con el objetivo de no incorporar costos en las pruebas, se generó una instancia Free Tier en Amazon Web Services, en la cual se instaló Ubuntu, el broker Mosquitto, el backend, la página Web y un servidor NGINX para que funcione como proxy reverso. También se contrató el servicio route 53 que permite personalizar las políticas de ruteo.
De esta manera, se puede acceder al sistema desde cualquier dispositivo móvil conectado a una red.
Los pasos para configurar el sistema backend en la instancia EC2 con ubuntu son:
\pagebreak

\begin{enumerate}


\item Loguearse en la instancia con ssh.
\item Setear la base temporal en Buenos Aires Argentina (esto es importante \\ ya que hay eventos de calendario que se deben realizar en el tiempo correspondiente a este huso horario). Esto se presenta en el código \ref{cod:Set DateTime}.\\

\begin{lstlisting}[label=cod:Set DateTime,caption=  Configuración de zona horaria.]
	sudo echo "America/Argentina/Buenos_Aires" | sudo tee /etc/timezone
	sudo dpkg-reconfigure --frontend noninteractive tzdata
	sudo reboot
\end{lstlisting}

\item Instalar Git:

	sudo apt-get install git
\item Instalar el broker Mosquitto:
	
	sudo apt-get install mosquitto
\item Configurar el brocker Mosquitto según la sección \ref{Broker Mosquitto}.
\item Instalar Docker y Docker-compose según el sitio \citep{WEBSITE:8}
\item Descargar el backend desde GitHub:

	git clone  https://github.com/gustavobastian/ServerNurse
\item Descargar desde \citep{WEBSITE:44} el archivo con la base de datos de ejemplo. Luego transferir a la instancia el archivo (utilizar el protocolo SCP)  y descomprimir dentro de la carpeta /db (ver \citep{WEBSITE:43}). Para descomprimir, dentro de la carpeta /db ejecutar:

sudo tar xvjf <dirección relativa del archivo comprimido>


\pagebreak
\item Crear el archivo de entorno (.env) según el código \ref{cod:Entorno_backend}:


\begin{lstlisting}[label=cod:Entorno_backend,caption=  Entorno del backend.]
TAG="v1.0.1"

##SECURITY
#secret pass
JWT_SECRET = <palabra secreta para las contrasenas>
#token expiration time
JWT_EXP_TIM = 120s


##MQTT CONFIGURATION
##setting por for using websockets
MQTT_CONNECTION = 'ws://<IP de broker>:<puerto de broker>'

##server
PORT_LOCAL    = 3000

##timezone

TZ = America/Argentina/Buenos_Aires 
\end{lstlisting}


\item Ejecutar Docker-compose up
\end{enumerate}

Los pasos para configurar la página web son:
\begin{itemize}
\item Compilar la página web: para ello, dentro del proyecto ejecutar:

''ionic build --prod''
\item Comprimir la carpeta www.
\item Loguearse con ssh en la instancia EC2.
\item Copiar el archivo comprimido en la instancia y descomprimirlo en ''/var/www/hmtl''.
\item Configurar nginx para redirigir a la página (ver apéndice \ref{AppendixB}).

\end{itemize}

\subsection{Generación/instalación de la aplicación móvil en Android}

Para que la aplicación pueda utilizarse en un dispositivo móvil, la misma debe ser compilada para ejecutarse en código nativo. Con esto se genera el código instalable .apk para Android y .ipa para IOS. 

Para poder descargar el archivo ejecutable a un dispositivo Android se debe generar el archivo .apk correspondiente. Para ello se deben ejecutar los siguientes pasos:

\begin{enumerate}

\item Generar el código con ionic capacitor:
''ionic cap build android''
\item Abrir Android Studio (ver apéndice \ref{AppendixC}).
\item Generar el bundle o en su defecto, conectar el celular en modo debug y descargar la aplicación. 
\end{enumerate}
\pagebreak


\subsection{Equipo simulador de llamadores}

Se desarrolló un dispositivo que simula los llamadores con un microcontrolador ESP32, una fuente y un teclado matricial. Presionando una tecla, se genera un evento en MQTT. El código y el esquemático del mismo se encuentra en \citep{WEBSITE:33} y se presentan algunos detalles del código en el anexo \ref{AppendixA}. Es importante mencionar que el dispositivo desarrollado posee la funcionalidad mínima, pero permite simular múltiples habitaciones (no fue pensado para un uso real).

\begin{figure}[ht]
	\centering
	\includegraphics[scale=.35]{./Figures/simulador.png}
	\caption{Simulador de llamadores.}
	\label{fig:Simulador de llamadores}
\end{figure}

\subsection{Resultados de las pruebas de integración}
\label{Resultados de las pruebas de integración}
El sistema permaneció funcionando de manera adecuada durante una semana completa, generando los eventos programados y permitiendo la interacción de dispositivos.

\pagebreak
\section{Contraste con los requerimientos}
\label{Contraste con los requerimientos}

En esta subsección se detalla el grado de cumplimiento de los requerimientos relevados en el plan de proyecto.
\begin{enumerate}
\item Requerimientos del servidor: 

1.1 debe tener instalado el broker mosquitto.

Verificación: Se muestra el funcionamiento del servidor utilizando la aplicación MQTT explorer.

\item Requerimientos de la base de datos: 

2.1 Debe poseer una base de datos relacional. 

2.2 Debe poseer las siguientes tablas: eventos, pacientes, médicos, enfermeras. 

2.2 La base de datos debe poseer datos cargados por default.

Verificación: Se muestra el contenido de la base de datos con la aplicación phpMyAdmin.

\item Requerimientos de la página web 

3.1 Debe ser cliente del broker MQTT. 

Verificación: Se presenta la configuración del broker. Se presenta el monitoreo de las habitaciones y los usuarios conectándose al broker.

3.2 La página debe poseer funciones de consulta o modificación de la base de datos.

Verificación: Se muestra como la página web permite cambiar contraseñas de usuarios y otros parámetros.

3.3 La página debe permitir observar la estadística de pacientes y enfermeras.

Verificación: Se muestra como la página web permite observar distintas gráficas.

3.4 La página debe contener acceso con usuario y contraseña para cada persona.


Verificación: Se muestra como la página web permite loguear los distintos usuarios.
 
(*) Cumplimiento parcial: solo permite loguear al usuario administrador.

\item Requerimientos de la aplicación móvil

4.1 y 4.2 La aplicación debe poseer tres modos de uso: médico, enfermera y sistema. 

Verificación: Se ingresa a la aplicación en los distintos modos.

4.3 La aplicación en modo enfermera debe permitir leer código QR.

Verificación: Se ingresa a la aplicación en modo enfermera y siguiendo los pasos se obtiene el código QR.

4.4 La aplicación en modo enfermera debe descargar información relevante del paciente.

Verificación: Se ingresa a la aplicación en modo enfermera y siguiendo los pasos se descarga las notas del paciente.

4.5 La aplicación en modo sistema debe mostrar las habitaciones sin atención, según una tabla de prioridades y en caso de igualdad de prioridades mostrar según un orden de llamada.

(*) Cumplimiento parcial: se muestra las habitaciones con una tabla de prioridades pero no según el orden de llamadas.

Verificación: Se ingresa a la aplicación en modo enfermera y siguiendo los pasos se descarga las notas del paciente.


4.6 El modo de usuario médico y el modo usuario enfermera deben poder enviar mensajes de texto o sonido.

Verificación: Se transmiten distintos audios entre participantes.

\item Requerimientos de la documentación

5.1 Documento con información relativa a la base de datos: detalles de la misma y API para acceder.

Verificación: Se observa la presencia de la información en el repositorio de GitHub.

5.2 Memoria del proyecto con diagramas de aplicación móvil y página web.

Verificación: se verifica con este documento.

\item Requerimientos de la integración del sistema

6.1 El sistema debe integrar el funcionamiento del servidor con la base de datos, aplicación web y aplicaciones móviles.

Verificación: se verificó el funcionamiento en un laboratorio con 4 usuarios y un simulador de 8 llamadores durante una semana.

(*) Cumplimiento parcial:
Validación: No se pudo validar el sistema en un nosocomio.

\item Requerimientos de la entrega del producto:

7.1 El Código fuente del servidor debe ser subido a dockerhub y compartido con la comunidad.

(*) Cumplimiento parcial:
Validación: Se subió el código fuente del servidor a GitHub.

7.2 El Código fuente de la aplicación debe ser subido a GitHub y compartido con la comunidad.

Validación: Se subió el código fuente del servidor a Github.

\end{enumerate}
