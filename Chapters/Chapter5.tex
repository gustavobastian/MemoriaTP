% Chapter Template

\chapter{Conclusiones} % Main chapter title

\label{Chapter5} % Change X to a consecutive number; for referencing this chapter elsewhere, use \ref{ChapterX}


%----------------------------------------------------------------------------------------

%----------------------------------------------------------------------------------------
%	SECTION 1
%----------------------------------------------------------------------------------------

\section{Notas sobre el sistema desarrollado }

%La idea de esta sección es resaltar cuáles son los principales aportes del trabajo realizado y cómo se podría continuar. Debe ser especialmente breve y concisa. Es buena idea usar un listado para enumerar los logros obtenidos.

%Algunas preguntas que pueden servir para completar este capítulo:

%\begin{itemize}
%\item ¿Cuál es el grado de cumplimiento de los requerimientos?
%\end{itemize}
%\item ¿Cuán fielmente se puedo seguir la planificación original (cronograma incluido)?
%\item ¿Se manifestó algunos de los riesgos identificados en la planificación? ¿Fue efectivo el plan de mitigación? ¿Se debió aplicar alguna otra acción no contemplada previamente?
%\item Si se debieron hacer modificaciones a lo planificado ¿Cuáles fueron las causas y los efectos?
%\item ¿Qué técnicas resultaron útiles para el desarrollo del proyecto y cuáles no tanto?
%\end{itemize}
En este capítulo se analiza el proyecto en su totalidad , el trabajo realizado y los pasos a seguir.



El principal aporte que presenta el proyecto consiste en el modelo que se propuso para el sistema de backend, con dos APIs totalmente separadas para interactuar con la base de datos. De esta manera, si se quiere escalar el proyecto , se pueden agregar funcionalidades de manera independiente a la página de administración y a la aplicación móvil. Otro aporte importante es el uso de MQTT para una aplicación de estas características con el modelo de subcripción a tópicos.

Los objetivos funcionales del desarrollo se cumplieron en un porcentaje muy alto, no así los planteados para la validación del sistema. El principal obstaculo para realizar una validación en un establecimiento radica en que legalmente no se puede utilizar en una situación real, ya que por la normativa nacional solo se puede permite el uso de software debidamente homologado en el país.  

\subsection{Cumplimiento de planificación original}

No se pudo seguir con la planificación original ya que se presentaron situaciones no contempladas en la misma. Se enumeran a continuación:
\begin{enumerate}
\item Agregado de funcionalidades necesarias pero no contempladas en un inicio.
\item Configuración de los dispositivos para la aplicación.

\end{enumerate}


\subsection{Gestión de riesgos}

El principal riesgo que se previó era la imposibilidad de desarrollar la aplicación para dispositivos IOS debido a la ausencia de un equipo en el cual desarrollar y otro para testear. Efectivamente se cumplió y no se pudo llegar a satisfacer ese requerimiento.

\subsection{Recursos aprendidos durante la carrera:}
En el desarrollo del trabajo se utilizaron las siguientes herramientas presentadas durante el trayecto de la carrera:

\begin{itemize}
\item Gestión de proyectos: se elaboró un plan de trabajo y el relevamiento de requerimientos.
\item Protocolos de Internet, Protocolos de IoT y Diseño de aplicaciones para IOT: se utilizaron los conceptos aprendidos sobre MQTT en estas materias.
\item Arquitectura de datos: se utilizó los conceptos sobre bases de datos enseñados en esta asignatura.
\item Ciberseguridad en Internet de las Cosas: se utilizaron los conceptos sobre seguridad enseñados en esta materia, principalmente en la página web de configuración.
\item Testing de sistemas de IoT: se utilizó la tecnica de testing del backend utilizando Postman enseñada en esta materia.
\item Diseño de páginas Web, Diseño de aplicaciones multiplataformas y Diseño de aplicaciones para Iot: se utilizaron los frameworks, las técnicas y los lenguajes de programación enseñados en estas materias.
\end{itemize}

%----------------------------------------------------------------------------------------
%	SECTION 2
%----------------------------------------------------------------------------------------
\section{Próximos pasos}

%Acá se indica cómo se podría continuar el trabajo más adelante.

Este trabajo sirve de base para la implementación de un sistema de gestión hospitalaria general. Los items que se pueden mejorar son:

\begin{itemize}
\item Generar el código para que la aplicación corra en IOS.
\item Utilizar una base de dato clave-valor para reemplazar la lista de usuarios en el monitoredo del backend.
\item Mejorar la seguridad utilizando TLS en el broker Mosquitto (por una razón de tiempo no se pudo desarrollar).
\item En el logeo de eventos se guardan números de referencia de usuario que realizó la acción y número de paciente. Se puede utilizar otra identificación suplementaria como ser nombre y apellido.
\item La estadística que se consulta a la base de datos es mínima. Se puede analizar otras cosas como ser duración promedio, pacientes con mayor tiempo de atención, etc.
\item Desarrollar un dispositivo llamador que se pueda utilizar en ámbitos hospitalarios.
\item Certificar los paquetes de software (aplicación móvil, backend y página web) de modo que se pueda comercializar.

\end{itemize}
