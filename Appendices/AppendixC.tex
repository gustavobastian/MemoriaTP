% Appendix C

\chapter{Compilación para aplicaciones Android} % Main appendix title

\label{AppendixC} % For referencing this appendix elsewhere, use \ref{AppendixA}

En esta sección se presentará brevemente los pasos para regenerar el código utilizado. Para el desarrollo de aplicaciones se consultó el documento \citep{BOOK:3} donde se menciona la importancia de la seguridad en las aplicaciones y en el resguardo de los datos.

Para poder compilar nativamente la aplicación es necesario poseer instalado Android Studio \citep{WEBSITE:41} y asignarle los permisos al software. Eso se logra incorporando componentes al android manifest, se observa en el código \ref{cod:AndroidMan}. Además es necesario incorporar algunas librerías al archivo variables.gradle y setear parametros en build.gradle. Para una descripción detallada de todo el proceso de generación de código ejecutable se puede recurrir a la documentación oficial \citep{WEBSITE:42};

La API utilizada es Android API:33.

El archivo Android Manifest describe la información esencial de la aplicación para que puedan ser interpretadas por las herramientas de generación de código ejecutable, el sistema operativo Android y Google Play.

Entre muchas otras cosas, el archivo de manifiesto debe declarar lo siguiente:
\begin{itemize}


\item El nombre del paquete de la aplicación, que normalmente coincide con el espacio de nombres de tu código. Las herramientas de compilación de Android usan esto para determinar la ubicación de las entidades de código cuando se compila el proyecto. Al empaquetar la aplicación, las herramientas de compilación sustituyen este valor por el ID de aplicación de los archivos de compilación de Gradle, que se utiliza como identificador único de la aplicación en el sistema y en Google Play. 
\item Los componentes de la aplicación, que incluyen todas las actividades, servicios, receptores de emisiones y proveedores de contenido. Cada componente debe definir propiedades básicas, como el nombre de su clase Kotlin o Java. También puede declarar capacidades, como las configuraciones de dispositivos que puede manejar, además de filtros de intents que describen cómo se puede iniciar el componente. 
\item Los permisos que necesita la aplicación para acceder a las partes protegidas del sistema o a otras aplicaciones. También declara cualquier permiso que otras aplicaciones deben tener si quieren acceder al contenido de esta aplicación. 
\end{itemize}

El archivo build.gradle de nivel superior, ubicado en el directorio raíz del proyecto, define dependencias que se aplican a todos los módulos de tu proyecto y en el archivo variables.gradle se indica la versión de APK que se utiliza. 

\begin{lstlisting}[label=cod:AndroidMan,caption=Modificaciones al Android Manifest.]  % Start your code-block

package="com.gabiot.Enfermera">
    <!-- Permissions -->

    <uses-permission android:name="android.permission.INTERNET" />
    <uses-permission android:name="android.permission.READ_EXTERNAL_STORAGE" />
    <uses-permission android:name="android.permission.WRITE_EXTERNAL_STORAGE" />
    <uses-permission android:name="android.permission.ACCESS_NETWORK_STATE" />
    <uses-permission android:name="android.permission.ACCESS_WIFI_STATE" />

    <uses-permission android:name="android.permission.WAKE_LOCK" />
    <uses-permission android:name="android.permission.READ_PHONE_STATE" />
    <uses-permission android:name="android.permission.CAMERA" />
    <uses-sdk tools:overrideLibrary="com.google.zxing.client.android" />
    <uses-permission android:name="android.permission.RECORD_AUDIO" />

    <application
        android:allowBackup="true"
        android:icon="@mipmap/ic_launcher"
        android:label="Enfermera"
        android:roundIcon="@mipmap/ic_launcher_round"
        android:supportsRtl="true"
        android:theme="@style/AppTheme"
        android:usesCleartextTraffic="true"
        android:hardwareAccelerated="true">
        <!-- Mqtt Service -->
        <service
            android:enabled="true"
            android:name="org.eclipse.paho.android.service.MqttService"
            android:exported="false"
        />
...

\end{lstlisting}


\begin{lstlisting}[label=cod:variablesGradle,caption= Archivo variables.gradle.]  

ext {
    minSdkVersion = 21
    compileSdkVersion = 33
    targetSdkVersion = 33
    androidxActivityVersion = '1.2.0'
    androidxAppCompatVersion = '1.2.0'
    androidxCoordinatorLayoutVersion = '1.1.0'
    androidxCoreVersion = '1.3.2'
    androidxFragmentVersion = '1.3.0'
    junitVersion = '4.13.1'
    androidxJunitVersion = '1.1.2'
    androidxEspressoCoreVersion = '3.3.0'
    cordovaAndroidVersion = '7.0.0'
}

\end{lstlisting}

\begin{lstlisting}[label=cod:buildGradle,caption= Archivo build.gradle(Module:android.app).]  
apply plugin: 'com.android.application'

android {
    //compileSdkVersion rootProject.ext.compileSdkVersion
    compileSdk 31

    defaultConfig {
        applicationId "com.gabiot.Enfermera"
        minSdkVersion rootProject.ext.minSdkVersion
        targetSdkVersion rootProject.ext.targetSdkVersion
        versionCode 1
        versionName "1.0"
        testInstrumentationRunner "androidx.test.runner.AndroidJUnitRunner"
        aaptOptions {
             // Files and dirs to omit from the packaged assets dir, modified to accommodate modern web apps.
           //   Default: https://android.googlesource.com/platform/frameworks/base/+/282e181b58cf72b6ca770dc7ca5f91f135444502/tools/aapt/AaptAssets.cpp#61
            ignoreAssetsPattern '!.svn:!.git:!.ds_store:!*.scc:.*:!CVS:!thumbs.db:!picasa.ini:!*~'
        }
    }
    buildTypes {
        release {
            minifyEnabled false
            proguardFiles getDefaultProguardFile('proguard-android.txt'), 'proguard-rules.pro'
        }
    }
}

repositories {
    flatDir{
        dirs '../capacitor-cordova-android-plugins/src/main/libs', 'libs'
    }

    /*maven {
        url "https://repo.eclipse.org/content/repositories/paho-releases/"
    }*/

    maven {
        url "https://repo.eclipse.org/content/repositories/paho-snapshots/"
    }

}

dependencies {
    implementation fileTree(include: ['*.jar'], dir: 'libs')
    //noinspection GradleDependency
    implementation "androidx.appcompat:appcompat:$androidxAppCompatVersion"

    implementation project(':capacitor-android')
    testImplementation "junit:junit:$junitVersion"
    //noinspection GradleDependency
    androidTestImplementation "androidx.test.ext:junit:$androidxJunitVersion"
    androidTestImplementation "androidx.test.espresso:espresso-core:$androidxEspressoCoreVersion"
    implementation project(':capacitor-cordova-android-plugins')
    api ('org.eclipse.paho:org.eclipse.paho.client.mqttv3:1.2.5')
    implementation 'com.google.android.material:material:1.5.0'

    implementation ('org.eclipse.paho:org.eclipse.paho.android.service:1.1.1')
}

apply from: 'capacitor.build.gradle'

try {
    def servicesJSON = file('google-services.json')
    if (servicesJSON.text) {
        apply plugin: 'com.google.gms.google-services'
    }
} catch(Exception e) {
    logger.info("google-services.json not found, google-services plugin not applied. Push Notifications won't work")
}
\end{lstlisting}


En Android Studio se puede realizar un debug de la aplicación.
Cuando se ejecuta la aplicación consulta por los permisos correspondientes.
