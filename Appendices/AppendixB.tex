% Appendix B

\chapter{Configuración de servidor Nginx} % Main appendix title

\label{AppendixB} % For referencing this appendix elsewhere, use \ref{AppendixA}

Nginx \citep{WEBSITE:36} es un servidor web liviano y de alta performance diseñado para soportar alto tráfico. Uno de sus casos de uso es ser un servidor proxy inverso. Además es soportado por Linux.

Un proxy inverso es un servidor que acepta todo el tráfico y lo reenvía hacia un recurso específico, como ser el frontend.

Nginx trabaja enfocado a eventos, lo que permite procesar solicitudes de forma asíncrona, ahorrando memoria y espacio.

Para instalar Nginx en una instancia AWS ejecutando Ubuntu (o en cualquier sistema linux), se utiliza el comando ''apt install nginx'' ( como usuario root en la consola).

Para la configuración de Nginx en el servidor se utilizan los siguientes archivos y directorios:
\begin{itemize}
\item /etc/nginx : directorio de configuración.
\item /etc/nginx/nginx.conf : archivo de configuración principal.
\item /etc/nginx/sites-available : directorio donde se pueden guardar bloques de servidor por sitio. Nginx no utiliza los archivos de configuración de este directorios a menos que esten vinculados a sites-enabled. La configuración se realiza en este directorio y luego se habilita estableciendo un vinculo con el otro directorio.
\item /etc/nginx/sites-enabled : directorio donde se guardan los bloques de servidor habilitados por sitios. 
\item /etc/nginx/snippets : directorio donde se pueden guardar fragmentos de configuración.
\end{itemize}

Los dominios utilizados en las pruebas de este trabajo son:

\begin{itemize}
\item Frontend: http://www.gabiot.com.ar:8024
\item Backend (generado por docker): http://www.gabiot.com.ar:8000
\item phpMyAdmin (generado por docker): http://www.gabiot.com.ar:8001
\item Broker MQTT (utilizando websockets): http://www.gabiot.com.ar:9001
\end{itemize}

\pagebreak
Dentro de la carpeta sites-available se guarda un archivo de nombre ''Nurse.com.conf'' con el contenido presentado en el código \ref{cod:archivo nurse}:

\begin{lstlisting}[label=cod:archivo nurse, caption=Archivo \\/etc/nginx/sites-available/Nurse.com.conf.]
server {
	listen 8024 default_server;
	root /var/www/html;
	index index.html;
}
\end{lstlisting}

En la carpeta sites-enabled se genera el enlace dinámico al archivo anterior mediante la instrucción:
\begin{itemize}
\item ''sudo ln -s /etc/nginx/sites-available/Nurse.com.conf /etc/nginx/sites-enabled/Nurse.com.conf ''
\end{itemize}

El archivo de la página se debe descomprimir dentro de ''/var/www/html'' y luego de ello reiniciar el servidor Nginx con los comandos:
\begin{itemize}
\item ''sudo systemctl stop nginx'' 
\item ''sudo systemctl start nginx''
\end{itemize}






